% change la fonte
\usepackage[default]{sourcesanspro}

% on insert TikZ et des sous-packages
% très utiles
\usepackage{tikz}
\usetikzlibrary{calc}
\usepackage{tikzpagenodes}

% modification des titres
\usepackage{titlesec}

% création de couleurs
\usepackage{color}

% entêtes et pieds de page
\usepackage{fancyhdr}
\definecolor{white}{rgb}{1,1,1}
\definecolor{myblue}{RGB}{126,136,150}

\titleformat{\chapter} % type
    {\Huge\color{white}} % style
    {\thechapter} % quoi afficher
    {0pt} % espacement
    {\begin{tikzpicture}[remember picture,overlay]
      \draw[fill=myblue] (current page.north west)
        -- (current page.north east)
        -- ++ (0, -8)
        -- ++ (-\paperwidth, -2)
        -- cycle;
    \end{tikzpicture}
    } % "before"

% on déplace un peu le titre vers le haut
\titlespacing*{\chapter}{-18pt}{0pt}{50pt}



% package "fancyhdr"
\pagestyle{fancyplain} % ou "fancy", testez
\fancyhf{} % vire le contenu de header et footer
\renewcommand{\headrulewidth}{0pt} % vire la barre de séparation


 

\chead{%
    \begin{tikzpicture}[remember picture, overlay]
    \node [yshift=-100, inner sep=0, outer sep=0]
        at (current page.north){\includegraphics[width=\paperwidth]{head.png}};
    \node[yshift=-40,rotate=5, text width=0.5\paperwidth]
        at (current page.north) {%
            \Huge
            \textbf{``} % un guillemet à l'envers
            \textcolor{myblue}{\leftmark} % le titre
            \textbf{''} % un guillemet à l'endroit
        };
    \end{tikzpicture}
}


\fancyfoot[CE]{%
\begin{tikzpicture}[remember picture, overlay]
    \draw [fill=myblue]
        (current page.south east)
        -- (current page.south west)
        -- ++ (0, 4)
        -- cycle;
    \node [yshift=28pt,xshift=50pt]
        at (current page.south west)
        {\color{white}\rightmark};
    \node [yshift=28pt]
        at (current page.south)
        {\color{white}\thepage};
\end{tikzpicture}
}


\fancyfoot[CO]{%
\begin{tikzpicture}[remember picture, overlay]
    \draw [fill=myblue]
        (current page.south west)
        -- (current page.south east)
        -- ++ (0, 2)
        -- cycle;
    \node [yshift=28pt, xshift=-50pt]
        at (current page.south east)
        {\color{white}\thepage};
\end{tikzpicture}
}
\usepackage{titling}

% je définis le titre
\title{Mon example de thème}
% l'auteur (moi)
\author{%
    Patrice FERLET\\
    metal3d@gmail.com
}

% et la page de titre
\renewcommand{\maketitle}{%
    \begin{titlepage}
    \begin{tikzpicture}[remember picture, overlay]

    \node [yshift=-230, inner sep=0, outer sep=0]
        at (current page.north)
        {\includegraphics[width=\paperwidth]{head.png}};

    \node [shape=rectangle,
            fill=myblue,
            anchor=west,
            minimum width=1.4\paperwidth,
            xshift=-0.7\paperwidth,
            inner sep=2cm,
            rotate=10]
        at (current page.center)
        {\Huge\textcolor{white}{\textbf{\thetitle}}};

    \node [yshift=120, text width=\paperwidth, align=center]
        at (current page.south)
        {\Huge\color{myblue}\theauthor};
    \end{tikzpicture}
    \end{titlepage}
}
\setlength{\parindent}{12pt}
\setlength{\parskip}{6pt plus 2pt minus 1pt}
\setlength{\emergencystretch}{3em}

\chead{%
\begin{tikzpicture}[remember picture, overlay]
    \node [yshift=-100, inner sep=0, outer sep=0] at (current page.north) {\includegraphics[width=\paperwidth]{head.png}};
    \node[yshift=-40,rotate=5, text width=0.5\paperwidth] at (current page.north) {%
    \ifnum\value{page}>2
        \Huge
        \textbf{``}
        \textcolor{myblue}{\leftmark}
        \textbf{''}
    \fi
    };
\end{tikzpicture}
}

