本课程包含两部分内容:热力学、统计物理。

\emphA{热力学(thermodynamics)}是一种自上而下(top-down)的研究方式,它是形而上的、唯象的。用一句话可以形容热力学:“知其然而不知其所以然”。它主要研究宏观物理量之间的关系。

对热力学有重大贡献的物理学家有Carnot、Joule、Clausius、Kelvin等。

\blankline

\emphA{统计物理(statistical Physics)}是一种自下而上(bottom-up)的研究方式,它是形而下的、微观的。

对统计物理有重大贡献的物理学家见表 \ref{TAB_PHYSICIST_IN_STATISTICAL_PHYSICS}。

\begin{table}[htb]
	\begin{tabular}{c|c}
		\hline
		\emphA{阶段} & \emphA{人物} \\
		\hline
		经典统计 & Maxwell、Boltzmann、Gibbs、Einstein等 \\
		量子概念 & Planck、Einstein、Fermi、Dirac、Pauli、Bose等 \\
		量子统计 & von Neumann、Landau、Kramers、Pauli等 \\
		\hline
	\end{tabular}
	\caption{对统计物理有重大贡献的物理学家}
	\label{TAB_PHYSICIST_IN_STATISTICAL_PHYSICS}
\end{table}
统计物理可分为两个阶段。1860年至1902年,人们主要研究近独立子体系(简单地说,就是理想气体);1902年以后,出现系综理论,开始研究凝聚态系统。

统计物理的基础可以概括为\emphB{等概率原理}。