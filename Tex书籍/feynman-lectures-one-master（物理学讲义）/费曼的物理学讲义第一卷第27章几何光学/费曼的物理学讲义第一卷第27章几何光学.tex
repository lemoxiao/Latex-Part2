% !Mode:: "TeX:UTF-8"%確保文檔utf-8編碼
%新加入的命令如下:addchtoc addsectoc reduline showendnotes hlabel
%新加入的环境如下:common-format  fig scalefig xverbatim

\documentclass[12pt,oneside]{book}
\newlength{\textpt}
\setlength{\textpt}{12pt}
\newif\ifphone
\phonefalse


\usepackage{myconfig}
\usepackage{mytitle}



\begin{document}
\frontmatter

\titlea{费曼的}
\titleb{物理学讲义}
\titlec{第一卷第二部分}
\author{费曼}
\authorinfo{理查德•费曼(Richard Phillips Feynman,1918年5月11日-1988年2月15日),美国物理学家。1965年诺贝尔物理奖得主。提出了费曼图、费曼规则和重整化的计算方法,这些是研究量子电动力学和粒子物理学的重要工具。}
\editor{德山书生}
\email{a358003542@gmail.com}
\editorinfo{编者:湖南常德人,纯手工输入,有些地方翻译优化了一下。}
\version{0.01}
\titleLC

\addchtoc{前言}
\chapter*{前言}
\begin{common-format}
源码在github上。\\
\href{https://github.com/a358003542/xelatex-guide-book}{https://github.com/a358003542/xelatex-guide-book}

英文官网:\href{http://www.feynmanlectures.caltech.edu/}{http://www.feynmanlectures.caltech.edu/}

%这里空一行。

\end{common-format}


\addchtoc{目录}
\setcounter{tocdepth}{2}
\tableofcontents

\begin{common-format}
\mainmatter

\setcounter{chapter}{26}
\chapter{几何光学}

\section{引言}
本章将用所谓\uwave{几何光学}近似来讨论上一章的概念对许多实际装置和仪器的一些初步应用。这种近似是许多光学系统和仪器在其具体设计中最有用的一种方法。几何光学要么十分简单,要么十分复杂。这样说的意思是,或者我们只是很肤浅地学习它,使得我们利用它的一些规则就能粗糙地设计仪器,而这些规则又十分简单,以致在这里根本没有必要去讲述它们,因为它们实际上是中学水平的内容;或者如果我们想要知道透镜以及类似器件的微小误差,则题材又太复杂,以致在这里讨论它显得太深!如果有人想解决一个在透镜设计方面实际而详细的问题,包括象差分析在内,那么奉劝他去阅读有关著作,要不就利用折射定律找出光线通过各个表面的轨迹(这就是本书所要讲的做法),并求出它们从哪里射出以及是否形成一个满意的象。有人说这太麻烦了,但今天借助计算机,这是解决问题的一个正确的方法。人们可以提出问题,并且很容易一条光线接着一条光线地进行计算。因而问题最终确实变得十分简单,也用不到什么新的原理。而且事实证明,不论是初等还是高等光学,它们的规则对于别的领域来说,很少有什么特色,所以没有什么特殊的理由要把这一题材讲得太深,但有一个重要的例外。

几何光学最高深和抽象的理论是由哈密顿完成的,结果证明,这种理论在力学中有很重要的应用。实际上它在力学中甚至比在光学中更为重要,所以我们把哈密顿理论作为高等分析力学课程的一部分放在高年级或研究班去讲。因此从某种意义上来讲,几何光学除了以自身为目的外对外贡献还是很少的\footnote{原译文为:“因而估计到几何光学除了为本身的目的外很少有贡献之后”,很是绕口。原英文为:“So, appreciating that geometrical optics contributes very little, except for its own sake”。},现在我们就在上一章所概括的原理的基础上对简单光学系统的基本性质进行讨论。

\begin{fig}{几何光学基本几何讨论}
\label{fig:几何光学基本几何讨论}
\end{fig}

为了进行讨论,就必须有一个几何公式,其内容如下:如果有一个高($ h $)很小而底边($ d $)很长的直角三角形,则斜边$ s $(我们在求两条不同路径之间的时间差时将用到它)比底边长(图27-1)。长多少呢?其差值$ \Delta = s - d $可以用许多方法求得。其中一种方法是这样:由图可见$ s^2 - d^2 = h^2 $,或$ (s - d)(s + d) = h^2 $,但 $ s - d = \Delta $,而$ s + d \approx 2s $,于是

\begin{equation}
\label{Eq:I:27:1}
\Delta \approx h^2/2s.
\end{equation}

这就是讨论曲面成象时所需要的全部几何学!



\section{球面的焦距}
我们所要讨论的第一个而且最简单的情况,是把两种折射率不同的介质分开的那种单折射面(图27-2)。我们把具有任意折射率的情况留给学生去做,因为最重要的往往是\uwave{概念}而不是特殊情况,并且这样的问题在任何情况下解决起来都是很简单的。所以我们假定光在左方的速率为$ 1 $,在右方的速率为$ 1/n $,这里$ n $是折射率。光在玻璃中传播较慢,要小一个因子$ n $。

\begin{fig}{单折射面聚焦}
\label{fig:单折射面聚焦}
\end{fig}

现在假定有一$ O $点在玻璃表面之间距离$ s $处,另一点$ O' $在玻璃之内距离$ s' $处。我们想这样设计一个曲面,使每条从$ O $点射到表面上任何一点$ P $处的光线经折射后都行进到$ O' $点。为了做到这一点,必须使表面具有这样的形状,使光从$ O $走到$ P $所花的时间,亦即距离$ OP $除以光的速率(这里光速为1),加上$ n \cdot O'P $,也就是光从$ P $走到$ O' $所花的时间,等于一个常数(与$ P $点无关)。这一条件为我们决定表面的形状提供了一个方程,它的解告诉我们此表面是一个非常复杂的四次曲面,学生如有兴趣可用解析几何来进行计算。但是如果计算一个对应于$ s \to \infty $的特殊情况,那么事情就比较简单,因为这时的曲面是一个二次曲面,我们对它比较熟悉。如果将这个曲面与当光来自无穷远时我们所求得的聚焦镜的抛物面进行比较,则是令人十分感兴趣的。

因而正确的表面不易制造,因为要把光从一点聚焦到另一点需要非常复杂的曲面。事实证明,我们在实践中一般并不去制造这种复杂的曲面,而作一妥协。我们不想把\uwave{所有}的光线都聚焦到一点,而是这样做,使得只有相当靠近$ OO' $轴的光线聚焦到一点。遗憾的是,离轴较远的光线即使想要聚焦到一点也会偏离,因为理想的表面很复杂,而我们只是用了一个在轴上具有适当曲率的球面来代替它的缘故。制造一个球面要比制造其他曲面容易得多,因此找出射到球面上的光线将会出现什么情况对我们是有用的,我们假定只有近轴的光线被完全聚焦。考虑轴的那些光线有时叫做\uwave{近轴光线},而我们所分析的就是近轴光线聚焦的条件。以后我们还将讨论不是所有光线都是近轴的情况下所导致的误差。

因此,假定$ P $靠近轴,我们做垂线$ PQ $,使$ PQ $之高为$ h $,并且暂且设想表面是一个通过$ P $的平面。在这种情况下,从$ O $到$ P $所需的时间将超过从$ O $到$ Q $的时间,同样,从$ P $到$ O' $的时间亦将超过从$ Q $到$ O' $的时间。但这正是玻璃所以必须弯曲的原因,因为所超过的总的时间必须由从$ V $到$ Q $所延迟的时间来补偿!现在沿路径$ OP $所\uwave{超过}的时间为$ h^2/2s $,而在$ PQ' $路径上所超过的时间为$ nh^2/2s' $。这后一个所超过的时间必须与沿$ VQ $所延迟的时间相抵消,但它与在真空中的不同,因为有介质存在。换句话说,光从$ V $行进到$ Q $的时间不是像它直接在空气中行进时一样,而是比之慢了$ n $倍,所以在这段距离内剩余的延迟时间为$ (n - 1)VQ $。但$ VQ $有多长?如果点$ C $为球心,$ R $为半径,那么由同一公式我们可以看到这段距离$ VQ $等于$ h^2/2R $,因此我们发现联系距离$ s $与$ s' $而又给出所需表面的曲率半径$ R $的规律为:

\begin{equation}
\label{Eq:I:27:2}
(h^2/2s) + (nh^2/2s') = (n-1)h^2/2R
\end{equation}

或\endnote{
这里的思路是走$ OP $折线相对于平面走直线的情况多了两个时间加量,时间加量为$ h^2/2s $和$ nh^2/2s' $,这个好理解。而在曲面情况下走直线相对于平面走直线的情况有一个时间减量和一个时间加量。就是在空气中少走了一个$VQ$,在玻璃中多走了一个$ VQ $。这样平面下走直线成了一个参照物,我们假设这个时间为$  t $,那么上式左边加上$  t $的含义就是曲面情况下走$ OP $所用的时间,类似右边含义也是如此,建立等式让这两个曲面下走折线和走直线的时间相等,从而得到公式。
下图是用来说明$ VQ $为什么等于$ h^2/2R $的:
\begin{fig}{单折射面聚焦解析}
\label{fig:单折射面聚焦解析}
\end{fig}
}

\begin{equation}
\label{Eq:I:27:3}
(1/s)+(n/s')=(n-1)/R.
\end{equation}

如果我们有一点$ O $及另一点$ O' $,并且想把光从$ O $点聚焦到$ O' $点,那么就可用此公式来计算所需表面的曲率半径$ R $。

现在有趣的是,结果表明:具有同样曲率半径$R$的同一透镜对于其他距离也能聚焦,也就是说,对于任何两个倒数之和(其中一个乘上$n$)为一常数的距离也能聚焦。因此,一个给定透镜(只要限于近轴光线)不仅能把光从$O$聚焦到$O'$,而且也能把光在无数对其他点之间聚焦,只要这些成对的点满足math
等于一个表征透镜特性的常数这个关系。






\section{透镜的焦距}
现在我们继续讨论另一种很实用的情况。我们所使用的大多数透镜具有两个表面,而不是只有一个表面。这将对事情产生什么影响呢?假定有两个不同曲率的表面,它们之间的空间充满着玻璃(图27-5),我们想研究从点$ O $向另一点$ O' $聚焦的问题,该怎么做呢?

\begin{fig}{双面透镜成象}
\label{fig:双面透镜成象}
\end{fig}

回答是:首先,对第一个表面应用式\ref{Eq:I:27:3}而不考虑第二个表面。这将告诉我们,从$ O $点发出的光看来好象是向另外某一点(比如说$ O' $)会聚的或者是从这一点散发出来的,完全依符号而定。现在我们来考虑一个新的问题,即在玻璃与光线在其中会聚到某一点$ O' $

\section{放大率}
到现在为止,我们只讨论了轴上点的聚焦作用。现在我们来讨论不完全在轴上而稍微离开它一点的物体的成象,这样可以使我们了解\uwave{放大率}的性质。当我们装置一个透镜把来自灯丝的光聚焦在屏上一“点”时,我们注意到,在屏上得到同一灯丝的“图象”,只是其大小比实际的灯丝大一些或小一些而已。这必然意味着灯丝上\uwave{每一点}发出的光都会聚到一焦点上。为了更好地理解这一点,我们来分析图27-7中所示的薄透镜系统。

\begin{fig}{薄透镜成象的几何图}
\label{fig:薄透镜成象的几何图}
\end{fig}

我们知道下列事实:
\begin{enumerate}
\renewcommand{\labelenumi}{(\arabic{enumi})}
\item 从一边射来的任一平行于轴的光线都朝另一边称为焦点的特殊点行进,这个点与透镜相距$ f $。
\item 任一从一边的焦点发出而到达透镜的光线,都在另一边平行于轴射出。
\end{enumerate}

这就是我们用几何方法建立公式\ref{Eq:I:27:12}所需要的全部知识,具体步骤如下:假定离焦点某一距离$ x $处有一物体,其高为$ y $。于是我们知道光线之中有一条光线(如$ PQ $)将经透镜偏折而通过另一边的焦点$ R $。如果现在这个透镜能完全使$ P $点聚焦的话,那么只要找出另外一条光线的走向,就能找出这个焦点在哪里,因为新的焦点应在两条光线再次相交的地方。因此我们只要设法找出另外\uwave{一}条光线实际方向,而我们记得平行的光线通过焦点,\uwave{反之亦然}:即通过焦点的光线将平行地射出!所以我们画出一条光线$ PT $通过$ U $。(诚然参与聚焦的实际光线可能比我们所画的两条光线的张角小得多,但它们画起来较为困难,所以我们假设能作这条光线。)既然它将平行射出,我们就画出$ TS $平行于$ XW $。交点$ S $就是所要求的点。这个点决定了象的正确位置和正确高度。我们把高度称为$ y' $,离焦点的距离称为$ x' $。现在我们可以导出一个透镜公式,应用相似三角形 $  PVU $和$  TXU $,得

\begin{equation}
\label{Eq:I:27:13}
\frac{y'}{f}=\frac{y}{x}.
\end{equation}

同样,从三角形SWR和QXR,得

\begin{equation}
\label{Eq:I:27:14}
\frac{y'}{x'}=\frac{y}{f}.
\end{equation}

由此上二式各解出$ y'/y $后,得

\begin{equation}
\label{Eq:I:27:15}
\frac{y'}{y}=\frac{x'}{f}=\frac{f}{x}.
\end{equation}

式\ref{Eq:I:27:15}是有名的透镜公式;其中包括了我们关于透镜所需要知道的一切:它告诉我们放大率$ y'/y $,用距离和焦距表示。它也把两个距离$ x $和$ x' $与$ f $联系起来:

\begin{equation}
\label{Eq:I:27:16}
xx'=f^2,
\end{equation}

这是一个用起来比\ref{Eq:I:27:12}简洁得多的形式。我们让同学自己去证明,若令$ s=x+f $,$ s'=x'+f $,则\ref{Eq:I:27:12}与\ref{Eq:I:27:16}相同。



\section{透镜组}



\section{象差}


\section{分辨本领}
另一个有趣的问题——一个对于所有光学仪器都很重要的技术问题——是它们的分辨本领有多大。如果我们制造一架显微镜,就想看清楚所观察的物体,举例来说,这意味着,如果我们正在观察一个两端都有斑点的细菌,那么我们就要做到在把它们放大时能看清楚有两个小点,人们也许会想,这只要把它们足够放大就行——我们总是可以再加上一个透镜,放大了又放大,而且凭着设计者的智慧,所有的球差和色差都可消除,因而没有理由说为什么不能不断地把象放大。所以显微镜的限度不在于不可能制造一个径向放大率大于2000倍的透镜。我们能够制造一个径向放大率为10,000倍的透镜系统,但由于几何光学的局限性,以及最短时间原理并非精确成立这一事实,我们\uwave{仍然}不能看清楚靠得太近的两个点。

要找出用以决定两个点应分得多开才能使它们的象看起来好像是分开的两个点的规则,可以结合不同光线所需的时间用一种很美妙的方法来叙述。假定现在不考虑象差,并设想对某一特殊点$P$来说(图27-9),所有从物到象$T$的光线所化的时间完全相同。(这是不确切的,因为它不是一个完善的系统,但那是另一个问题。)现在选取附近另一个点$P'$,并问其象是否能与$T$分清楚。换句话说,我们是否能辨认出它们之间的差别。按照几何光学,当然应该有两个点象,但我们看到的可能是一个比较模糊的斑点,以致无从辨认出那里有两个点。第二个点聚焦在与第一个点显著不同的另一个地方的条件是,通过透镜大开孔的两个边缘的极端光线$P'ST$与$P'RT$,它们从一端行进到另一端所化的两个时间,必须与从两个可能的物点到同一给定的象点所化的时间\uwave{不}同。为什么?因为,如果时间相同,两个物点当然都\uwave{聚焦}在同一点上。所以这两个时间不应相同。但它们必须相差多少,才可以说两个物点\uwave{不}都聚焦在同一点上,以致我们可以分清两个象点?对任何光学仪器,其分辨本领的一般规则是这样的:两个不同的点源,只有当一个点源聚焦在某一点,而从另一点源发出的两条极端光线到达这一点所化的时间与到达它自己的实际象点相比,相差大于一个周期时,才能被分辨。亦即顶端光线与底边光线到达非正确焦点的时间差别必须大于某一数值,这个数值才近似地等于光的振动周期:



\section{注释}
\showendnotes



%这里空一行

\end{common-format}
\end{document}



