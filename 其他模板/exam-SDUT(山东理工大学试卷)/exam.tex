\documentclass[12pt,oneside,UTF8]{exam}

\def\exCourse{《空间解析几何》期末考试}
\def\exData{2018---2019~学年第~1~学期}
\def\exType{A}

\begin{document}\zihao{4}

\exBegin{\exCourse}{\exData}{\exType}%用于生成试卷表头
\exScore%用于生成打分表

\exBig{选择题}{10}{每题2分,}

\exChA{宋宁老师的女儿的生日是哪一天?}
{2019年9月20日}{2019年8月20日}{2018年9月20日}{2018年8月20日}

\exChB{以下对宋宁老师女儿的描述不正确的是:}
{美如天仙}{身体强健}{胆大包天}{特别乖}

\exChC{宋宁老师的女儿的生日是哪一天?}
{2019年9月20日}{2019年8月20日}{2018年9月20日}{2018年8月20日}

\exBig{填空题}{10}{每题2分,}

\exFill{宋宁老师的生日是\underline{~~~~~~~~~~~~~~}.}

\exFill{宋宁老师的祖籍是\underline{~~~~~~~~~~~~~~}.}

\exNewPage%建立新页面

\exBig{计算题}{40}{每题10分,}

\exProof{求$n$阶完全图的交叉数${\rm cr}(K_n)$}{40}

\exProof{求$n$阶完全图的交叉数${\rm cr}(K_n)$}{40}

\exProof{求$n$阶完全图的交叉数${\rm cr}(K_n)$}{40}

\exProof{求$n$阶完全图的交叉数${\rm cr}(K_n)$}{40}

\exNewPage%建立新页面

\exBig{证明题}{40}{每题20分,}

\exProof{求证:任意无割边的可平面图的点色数不超过$4$}{70}

\exProof{求证:任意无割边的可平面图的点色数不超过$4$}{70}

\end{document}