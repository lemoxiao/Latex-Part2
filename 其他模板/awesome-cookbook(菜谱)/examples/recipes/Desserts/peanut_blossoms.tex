\recipe[style=style2]{Peanut Blossoms}

\info[servings=8,
		time = 100, 
		energy = 304, 
		urlsource = http://allrecipes.com/recipe/9920/peanut-blossoms-ii/]{}

\begin{ingredients}
	\ingredient{230}{g}{shortening}
	\ingredient{230}{g}{peanut butter}
	\ingredient{230}{g}{brown sugar}
	\ingredient{230}{g}{white sugar}
	\ingredient{2}{}{eggs}
	\ingredient{20}{g}{flour}
	\ingredient{240}{ml}{milk}
	\ingredient{10}{g}{vanilla extract}
	\ingredient{790}{g}{flour}
	\ingredient{10}{g}{baking soda}
	\ingredient{5}{g}{salt}
	\ingredient{115}{g}{white sugar (decoration)}
	\ingredient{255}{g}{milk chocolate candy kisses}
\end{ingredients}

\begin{preparation}
	\step Preheat oven to 375 degrees F (190 degrees C). Grease cookie sheets.
	
	\step In a large bowl, cream together the shortening, peanut butter, brown sugar, and 1 cup white sugar until smooth. Beat in the eggs one at a time, and stir in the milk and vanilla. Combine the flour, baking soda, and salt; stir into the peanut butter mixture until well blended. Shape tablespoonfuls of dough into balls, and roll in remaining white sugar. Place cookies 2 inches apart on the prepared cookie sheets.
	
	\step Bake for 10 to12 minutes in the preheated oven. Remove from oven, and immediately press a chocolate kiss into each cookie. Allow to cool completely; the kiss will harden as it cools.
\end{preparation}


\begin{notes}
	\note{This recipe uses a \texttt{style2} header and \texttt{ingredients} environment.}
	\note{There is also a picture, using \texttt{recipefigure} with the \texttt{clip} style. This enlarge the image to the set with and crop it afterwards to the set height. For this example we have used \texttt{0.33*textheight} and \texttt{textwidth}.}
\end{notes}


\recipefigure[style=crop,height=0.33\textheight, width=\textwidth]{peanutBlossoms.jpg}
