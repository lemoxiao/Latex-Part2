\recipe[style=style2, startleft=true]{Zucchini Brownies}

\info[servings=24,
		servingstext = portions,
		time = 45, 
		energy = 209,
		urlsource = http://allrecipes.com/recipe/25112/zucchini-brownies/]{}

\begin{ingredients}
	\ingredient{119}{ml}{vegetable oil}
	\ingredient{340}{g}{white sugar}
	\ingredient{12.5}{g}{vanilla extract}
	\ingredient{450}{g}{flour}
	\ingredient{119}{g}{cocoa powder}
	\ingredient{7.5}{g}{baking soda}
	\ingredient{5}{g}{salt}
	\ingredient{450}{g}{succhini}
	\ingredient{119}{g}{walnuts}
	\ingredient{60}{g}{margarine}
	\ingredient{450}{g}{confectioners' sugar}
	\ingredient{60}{ml}{milk}
\end{ingredients}

\begin{preparation}
	\step Preheat oven to 350 degrees F (175 degrees C). Grease and flour a 9x13 inch baking pan.
	
	\step In a large bowl, mix together the oil, sugar and 2 teaspoons vanilla until well blended. Combine the flour, 1/2 cup cocoa, baking soda and salt; stir into the sugar mixture. Fold in the zucchini and walnuts. Spread evenly into the prepared pan.
	
	\step Bake for 25 to 30 minutes in the preheated oven, until brownies spring back when gently touched. To make the frosting, melt together the 6 tablespoons of cocoa and margarine; set aside to cool. In a medium bowl, blend together the confectioners' sugar, milk and 1/2 teaspoon vanilla. Stir in the cocoa mixture. Spread over cooled brownies before cutting into squares.
\end{preparation}


\begin{notes}
	\note{This recipe uses a \texttt{style2} header and \texttt{ingredients} environment. This recipe will always start on the left page because it uses the option \texttt{startleft}, this may cause a blank page in front of the recipe.}
	\note{There is also a picture, using \texttt{recipefigure} with the \texttt{fullpage4} style.}
\end{notes}


\recipefigure[style=fullpage4, fig2=brownie.jpg, fig3=brownie.jpg, fig4=brownie.jpg]{brownie.jpg}
