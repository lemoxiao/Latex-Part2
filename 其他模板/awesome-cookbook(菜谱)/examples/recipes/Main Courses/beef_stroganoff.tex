\recipe[style=style4]{Beef Stroganoff}

\info[servings=8,
		time = 1:40,
		timeunit = h:min,
		energy = 304, 
		urlsource = http://allrecipes.com/recipe/25202/beef-stroganoff-iii/]{}

\begin{ingredients}
	\ingredient{450}{g}{beef chuck roast}
	\ingredient{2.5}{g}{salt}
	\ingredient{2.5}{g}{black pepper}
	\ingredient{115}{g}{butter}
	\ingredient{4}{}{green onions}
	\ingredient{20}{g}{flour}
	\ingredient{1}{}{beef bronth (condensed)}
	\ingredient{5}{}{mustard}
	\ingredient{170}{g}{mushrooms}
	\ingredient{80}{ml}{sour cream}
	\ingredient{80}{ml}{white wine}
\end{ingredients}

\begin{preparation}
	\step Remove any fat and gristle from the roast and cut into strips 1/2 inch thick by 2 inches long. Season with 1/2 teaspoon of both salt and pepper.
	
	\step In a large skillet over medium heat, melt the butter and brown the beef strips quickly, then push the beef strips off to one side. Add the onions and cook slowly for 3 to 5 minutes, then push to the side with the beef strips.
	
	\step Stir the flour into the juices on the empty side of the pan. Pour in beef broth and bring to a boil, stirring constantly. Lower the heat and stir in mustard. Cover and simmer for 1 hour or until the meat is tender.
	
	\step Five minutes before serving, stir in the mushrooms, sour cream, and white wine. Heat briefly then salt and pepper to taste.
\end{preparation}


\begin{notes}
	\note{This recipe uses a \texttt{style4} header and \texttt{ingredients} environment.}
	\note{There is also a picture, using \texttt{recipefigure} with the \texttt{clip} style. This enlarge the image to the set with and crop it afterwards to the set height. For this example we have used \texttt{textheight} and \texttt{textwidth}. It does not automatically go to a new page.}
\end{notes}


\recipefigure[style=crop,height=\textheight, width=\textwidth]{beefStroganoff.jpg}