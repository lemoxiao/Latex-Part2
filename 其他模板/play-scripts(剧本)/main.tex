% This templete is obtained from http://kevindonnelly.org.uk/resources/playscript.tex

\documentclass[11pt,a4paper,oneside]{memoir}  % http://www.ctan.org/tex-archive/macros/latex/contrib/memoir
\usepackage[english]{babel}
\usepackage[utf8]{inputenc}
\usepackage{enumitem}  % http://www.ctan.org/tex-archive/macros/latex/contrib/enumitem
\usepackage{ctex}

\newlength{\drop}  % Without this, the title page will not compile correctly.
% To avoid using drop, see: http://wiki.lyx.org/LyX/UsingMemoirInLyX

\chapterstyle{demo2}  % See p92 of the Memoir manual.

\pagestyle{myheadings}

\setlength{\parindent}{0pt}

\renewcommand{\printtoctitle}[1]{\centering\Large\bfseries Acts}  % Set the title of the contents page.
% \renewcommand{\printtoctitle}[1]  % Remove the title from the contents page entirely.

\pagenumbering{gobble} % Remove page numbers until told otherwise.

% Various title pages may be used with the memoir package.  The one below is from ``Some Examples of Title Pages'' (Peter Wilson) at http://www.ctan.org/tex-archive/info/latex-samples/TitlePages.

% Set up the title page.
\newcommand*{\titleGM}{\begingroup% Gentle Madness title page style
  \drop = 0.1\textheight
  \vspace*{\baselineskip}
  \vfill
  \hbox{%
    \hspace*{0.2\textwidth}%
    \rule{1pt}{\textheight}
    \hspace*{0.05\textwidth}%
    \parbox[b]{0.75\textwidth}{
      \vbox{%
        % Main title of the play
        \vspace{\drop}{\noindent\HUGE\bfseries The Rivals}\\
        %\vspace{\drop}{\noindent\HUGE\bfseries Title of the play \\[0.5\baselineskip] over two lines}\\
        % Subtitle of the play
        [2\baselineskip]{\huge\itshape A Comedy}\\
        %[2\baselineskip]{\Large\itshape Subtitle of the play \\[0.5\baselineskip] over two lines}\\
        % Author of the play
        [4\baselineskip]{\Large Richard Brinsley Sheridan}\par\vspace{0.5\textheight}
        %[4\baselineskip]{\Large First Author \\[0.5\baselineskip] Second Author \\[0.5\baselineskip] Third Author \\}\par\vspace{0.5\textheight}
        % Publisher and year of publication
        {\noindent \textbf{John Wilkie} \\[0.5\baselineskip] \textbf{1775}}\\
        [\baselineskip]
      }% end of vbox
    }% end of parbox
  }% end of hbox
  \vfill
  \null
  \endgroup}

\begin{document}
  
  % Print out the title page.
  \titleGM
  
  \pagenumbering{roman} % Start numbering pages with Roman numerals (for the front matter).
  
  % Print out the contents page, listing the acts of the play.
  % You will need to run pdflatex twice before the page numbers show up.
  \tableofcontents*
  \clearpage
  
  % Print out the characters page, listing the dramatis personae
  % The starred form of \chapter prevents a chapter number (eg ``One'', ``Two'') being printed before each chapter title (eg ``Characters'', ``Act 1'').
  \chapter*{CHARACTERS}
  \begin{center}  % Centre the list of characters.  Comment out this line and \end{}center if centring is not desired.
    \textbf{Sir Anthony Absolute}, a rich nobleman \\
    \textbf{Faulkland}, a friend of Sir Anthony's son Jack \\
    \textbf{Lucius O'Trigger}, an Irish nobleman \\
    \textbf{Lydia Languish}, a teenage heiress \\
    \textbf{Mrs Malaprop}, Lydia's guardian \\
    \vskip 1cm
    
    \textbf{Scene}: Bath.
    \textbf{Time of action}: Within one day.
  \end{center}
  
  % Print out a page with any additional authorial comments, notes on staging, or whatever.
  \chapter*{PREFACE, NOTES, WHATEVER}
  
  % Set up a description list to hold the paragraphs.  Increase the space between the list items, and set the left margin to 0.20cm
  \begin{description}[itemsep=1ex,leftmargin=0.20cm]
    
    % Precede each paragraph with an empty \item[].
    \item[] The preface to a play seems generally to be considered as a kind of closet-prologue in which the author solicits that indulgence from the reader which he had before experienced from the audience.
    
    \item[] I need scarcely add that the circumstance alluded to was the withdrawing of the piece to remove those imprefections which were too obvious to escape reprehension.
    
    \item[] With regard to some particular passages which seemed generally disliked, I confess that if I felt any emotion of surprise at the disapprobation, it was not that they were disapproved of, but that I had not before perceived that they deserved it.
    
  \end{description}
  
  \clearpage
  \pagenumbering{arabic}  % Start numbering pages with Arabic numerals (for the text of the play).
  
  % Generate a running header with the title of the play.
  \markright{\textsc{Title of the play}}
  
  %%%%%%%%%%%%%%%%
  \chapter*{ACT 1}
  %%%%%%%%%%%%%%%%
  % The \chapter* will prevent the the chapters (Acts) being listed in the table of contents, so we need to add them manually.
  \addcontentsline{toc}{chapter}{Act 1}
  
  % The starred form of \section prevents a section number (eg ``1.1'', ``2.3'') being printed before each section title (eg ``Scene 1'', ``Scene 2'').
  \section*{\textit{SCENE 1}}
  %\section*{\hfill\textit{SCENE 1}}  % Use this line instead if you want the Scene 1 heading shifted to the right edge of the page.
  
  % Set up a description list to hold the dialogue of the scene.  Increase the space between the list items, and set the left margin to 1cm.
  \begin{description}[itemsep=1ex,leftmargin=1cm]
    
    % Where the scene or act begins with stage directions:
    \item[] \hfill \\
    \textit{A street in Bath.  Coachman crosses the stage. Enter Fag, looking after him.}
    
    % Where the scene or act begins with dialogue, and no stage directions:
    %\item[] \hfill
    
    % Wrap each character's name in \item[].  Wrap in-dialogue directions in \textit{}
    \item[FAG] \textit{(calls)} What!  Thomas! \textit{(aside)} Sure 'tis he! \textit{(calls)} What!  Thomas!  Thomas!
    
    \item[THOMAS] Hey!  Od's life!  Mr Fag!  Give me your hand, my old fellow-servant!
    
    \item[FAG] Excuse my glove, Thomas.
    
    \item[THOMAS] Sure, and the postillion be all come!
    
    % Close the description list at the end of the scene.
  \end{description}
  \vskip 1cm  % Put a bit of space between this and the next scene heading.
  
  \section*{\textit{SCENE 2}}
  \begin{description}[itemsep=1ex,leftmargin=1cm]
    
    % Where the scene or act begins with dialogue, and no stage directions:
    \item[] \hfill
    
    \item[LYDIA] Heigh-ho!  Did you inquire for \textit{The Delicate Distress}?
    
    \item[LUCY] Or \textit{The Memoirs of Lady Woodford}? Yes indeed, ma'am.  I asked everywhere for it; and I might have brought it from Mr Frederick's, but Lady Slattern Lounger, who had just sent it home, had so spoiled and dog's-eared it, it wa'n't fit for a Christian to read.
    
  \end{description}
  \vskip 1cm
  
  
  %%%%%%%%%%%%%%%%
  \chapter*{ACT 2}
  %%%%%%%%%%%%%%%%
  \addcontentsline{toc}{chapter}{Act 2}
  
  \section*{\hfill\textit{SCENE 1}}  % The \hfill will shift the scene heading to the right edge of the page.
  \begin{description}[itemsep=1ex,leftmargin=1cm]
    \setlength{\parskip}{5pt}
    
    \item[] \hfill \\
    \textit{Absolute's lodgings.}
    
    \item[FAG] Sir, while I was there, Sir Anthony came in.  I told him you had sent me to inquire after his health, and to know if he was at leisure to see you.
    
    \item[ABSOLUTE] And what did he say on hearing I was at Bath?
    
    \item[FAG] Sir, in my life I never saw an elderly gentleman more astonished!
    
  \end{description}
  \vskip 1cm
  
  \section*{\hfill\textit{SCENE 2}}
  \begin{description}[itemsep=1ex,leftmargin=1cm]
    \setlength{\parskip}{5pt}
    
    \item[] \hfill \\
    \textit{The North Parade.}
    
    \item[LUCY] So, I shall have another rival to add to my mistress's list, Captain Absolute.
    
    \item[SIR LUCIUS] Hah! My little embassadress!
    
    \item[LUCY] \textit{(speaking simply)} O gemini!  And I have been waiting for your worship here on the North.
    
  \end{description}
  \vskip 1cm
  
  
\end{document}