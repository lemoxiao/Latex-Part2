
%\problem[бодолтын хэсгийн өндрийн хэмжээ]{бодлогын өгүүлбэр}{оноо}{бодолт ба хариу}

\problem[2.5]{
$(X,Y)$ вектор санамсаргүй хувьсагчийн хамтын тархалтын хууль дараах хүснэгтээр өгөгдөв.
\begin{center}
\begin{tabular}{r|rrr}
  & \multicolumn{3}{c}{$Y$} \\
  $X$  & $-1$ & 0 & 1 \\
  \hline
  $-1$ & 0.2 & 0.2 & 0.1 \\
  0 & 0 & 0.1 & 0.2 \\
  1 & 0 & 0 & 0.2 \\
\end{tabular}
\end{center}
$X$ ба $Y$ санамсаргүй хувьсагчид хамааралтай эсэхийг тогтоо.
}{
5
}{
хамааралтай
}

\begin{lrbox}{\lstListing}
\begin{lstlisting}[language=python]
import random, math
Lambda = float( raw_input("Lambda = ") )
print -1.0 * math.log( random.random() ) / Lambda
\end{lstlisting}
\end{lrbox}

\problem[3]{
Дор ямар санамсаргүй хувьсагчийг загварчилсан байна вэ? \printlisting Мөн энд ашигласан томъёоны гаргалгааг хийж гүйцэтгэ.
}{
5
}{

}

%\question[хариултын хэсгийн өндрийн хэмжээ]{асуулт}{оноо}{хариу}

\question{
$X$ ба $Y$ хамааралгүй байг. Тэгвэл $E(XY)=?$
}{
3
}{
$EX\cdot EY$
}

%\stest{асуулт}{оноо}{сонголт1//[+]зөвсонголт//сонголт3}

\stest{
Моод ямар төрлийн тоон үзүүлэлт вэ?
}{
2
}{
[+]Төвийн//Хазайлтын//Тархалтын хэлбэрийн//Хамаарлын
}

%\ptest{асуулт \ehide{томъёо} асуулт \ehide{текст, олон мөр дамнахыг дэмжинэ} асуулт}{оноо}

\ptest{
$EX=2$ ба $EY=1$ байв. Тэгвэл $E(2X+Y)=\ehide{2EX}+EY$.
}{
3
}

\ptest{
Характеристик функцийн ашиглан тархалтын хуулийг олохдоо характеристик функцийн \thide{урвалтын томъёо}г хэрэглэдэг.
}{
3
}
