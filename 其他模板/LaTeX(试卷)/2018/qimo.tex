\documentclass{BHCexam}

\begin{document}
	\biaoti{2018期末}
	\fubiaoti{}
	\maketitle
	\begin{questions}
		\qs 若$ \log_2a+\log_{\frac{1}{2}}b=2 $,则有\xx
		\onech{$ a=2b$}{$b=2a$}{$ a=4b$}{$ b=4a$}
		\qs 已知直线$ x-y+m=0 $与圆$ O:x^2+y^2=1 $相交于$ A,B $两点,且$ \triangle OAB $为正三角形,则实数$ m $的值为\xx
		\onech{$ \dfrac{\sqrt{3}}{2}$}{$ \dfrac{\sqrt{6}}{2}$}{$ \dfrac{\sqrt{3}}{2}\text{或}-\dfrac{\sqrt{3}}{2}$}{$\dfrac{\sqrt{6}}{2}\text{或}-\dfrac{\sqrt{6}}{2} $}

		\qs 从编号分别为$ 1,2,3,4,5,6 $的六个大小完全相同的小球中,随机取出三个小球,则恰有两个小球编号相邻的概率是\xx
		\onech{$ \dfrac{1}{5}$}{$ \dfrac{2}{5}$}{$ \dfrac{3}{5}$}{$ \dfrac{4}{5}$}
		\qs 在$\triangle ABC$中,$ AB=AC=1, \ D$是$ AC $边的中点,则$ \vv{BD}\bm{\cdot}\vv{CD} $的取值范围是\xx
		\onech{$ \left(-\dfrac{3}{4},\dfrac{1}{4}\right)$}{$ \left(-\infty,\dfrac{1}{4}\right)$}{$ \left(-\dfrac{3}{4},+\infty\right)$}{$ \left(\dfrac{1}{4},\dfrac{3}{4}\right)$}
		\qs 已知$ M $为曲线$ C:\Bigg\{\begin{aligned}
		&	x=3+\cos\theta,\\
		&	y=\sin\theta 
		\end{aligned}\left(\theta\text{为参数}\right)$上的动点,设$ O $为原点,则$ \abs{OM} $的最大值是\xx
		\onech{$ 1$}{$ 2$}{$ 3$}{$ 4$}
		\qs 设$ \bm{a,b} $是非零向量,且$ \bm{a,b} $不共线,则“$ \abs{\bm{a}}=\abs{\bm{b}} $”是“$\abs{\bm{a+2b}}=\abs{\bm{2a+b}}$”的\xx
		\twoch{充分而不必要条件}{必要而不充分条件}{充分必要条件}{既不充分也不必要条件}
		\qs 函数$f(x)=2\sin\left(\omega x+\varphi\right)\left(\omega>0,\abs{\varphi}<\dfrac{\pi}{2}\right)$的部分图象如图所示,则$ \omega,\varphi $的值分别是\xx
		\begin{center}
			\begin{tikzpicture}[>=latex]
			\tikzmath{
			\a = 5*pi/12;
			\b=11*pi/12;
		}
			\draw[->](-1,0)--(4,0) node[below](x){$x$};
			\draw[->](0,-2.3)--(0,2.3) node[left](y){$y$};
			\node[below left](O) at(0,0){$\small O$};
			\draw[domain=0:pi,samples=1000] plot (\x,{2*sin((2*(\x)-pi/3) r)}); 
			\draw[dashed] (0,2)node[left](a){$2$}-|(\a,0)node[below](a1){$\dfrac{5\pi}{12}$} ;;
			\draw[dashed](0,-2)node[left](b){$-2$}-|(\b,0)node[above](b1){$\dfrac{11\pi}{12}$} ;
%			\draw[dashed] (0,2)-|($(5*pi/12,0)$);
			\end{tikzpicture}
		\end{center}
		\onech{$ 2,-\dfrac{\pi}{3}$}{$2,-\dfrac{\pi}{6} $}{$4,-\dfrac{\pi}{6} $}{$ 4,\dfrac{\pi}{3}$}
		\qs 以角$ \theta $的顶点为坐标原点,始边为$x$轴的非负半轴,建立平面直角坐标系,角$ \theta $终点过点$ P(2,4) $,则$ \tan\left(\theta+\dfrac{\pi}{4}\right)= $\xx
		\onech{$ -\dfrac{1}{3}$}{$ -3$}{$ \dfrac{1}{3}$}{$ 3$}
		\qs 实数$ x,y $满足$\begin{dcases}
		x-1\ge 0,\\
		x+y-1\ge 0,\\
		x-y+1\ge 0.
		\end{dcases}$则$ 2x-y $的取值范围是\xx
		\onech{$ \left[0,2\right]$}{$ \left(-\infty,0\right]$}{$ \left[-1,2\right]$}{$ \left[0,+\infty\right)$}
	
		\qs 已知函数$f(x)=\sin\left(x+\varphi\right)$的图象记为曲线$ C $,则“$f(0)=f\left(\pi\right) $”是“曲线$ C $关于直线$ x=\dfrac{\pi}{2} $对称”的\xx
		\twoch{充分而不必要条件}{必要而不充分条件}{充分必要条件}{既不充分也不必要条件}
		\qs “$ m>10 $”是“方程$ \dfrac{x^2}{m-10}+\dfrac{y^2}{m-8}=1 $表示双曲线”的\xx
		\twoch{充分而不必要条件}{必要而不充分条件}{充分必要条件}{既不充分也不必要条件}
		\qs 已知点$ F $为抛物线$ C:y^2=2px(p>0) $的焦点,点$ K $为$ F $关于原点的对称点,点$ M $在抛物线$ C $上,则下列说法错误的是\xx
		\fourch{使得$ \triangle MFK$为等腰三角形的点$ M $有且仅有$ 4 $个}{使得$ \triangle MFK$为直角三角形的点$ M $有且仅有$ 4 $个}{使得$\angle MKF=\dfrac{\pi}{4} $的点$ M $有且仅有$ 4 $个}{使得$\angle MKF=\dfrac{\pi}{6} $的点$ M $有且仅有$ 4 $个}

		%海淀文第8题
		\qs 已知正方体$ABCD-A_1B_1C_1D_1$的棱长为$ 2 $,$ M,N $分别是棱$ BC,\ C_1D_1 $的中点,点$ P $在平面$ A_1B_1C_1D_1 $内,点$ Q $在线段$ A_1N $上.若$ PM=\sqrt{5} $,则$ PQ $长度的最小值是\xx
		\begin{center}
		\swht{
		\coordinate[label=above right:\footnotesize$D$](D) at(0,0);
		\coordinate[label=below left:\footnotesize$A$](A) at(6,0);
		\coordinate[label=right:\footnotesize$C$](C) at(0,4);
		\coordinate[label=below:\footnotesize$B$](B) at(6,4);
		\foreach \i in {A,D}
		\draw (\i)--+(0,0,3) node[coordinate,label=left:\footnotesize$\i_1$](\i_1) {$\i_1$};
			\foreach \j in {B,C}
		\draw (\j)--+(0,0,3) node[coordinate,label=right:\footnotesize$\j_1$](\j_1) {$\j_1$};
		\coordinate[label=\footnotesize$N$](N) at($(C_1)!0.5!(D_1)$);
		\coordinate[label=below:\footnotesize$Q$](Q) at($(A_1)!0.7!(N)$);
		\coordinate[label=above:\footnotesize$P$] (P) at(4,3.5,3);
		\coordinate[label=right:\footnotesize$M$](M) at($(B)!0.5!(C)$);
		\draw(P)--(Q);
		\draw[dashed] (A)--(D)--(C);
		\draw (A)--(B)--(C);
		\draw(A_1)--(B_1)--(C_1)--(D_1)--cycle;
		\draw(A_1)--(N);
		\draw[densely dashed] (P)--(M);
	}	
		\end{center}
	\vspace{-1.2em}
		\onech{$ \sqrt{2}-1$}{$ \sqrt{2}$}{$ \dfrac{3\sqrt{5}}{5}-1$}{$ \dfrac{3\sqrt{5}}{5}$}
				\qs 现有$ n $个小球,甲、乙两位同学轮流且不放回抓球,每次最少抓$ 1 $个球,最多抓$ 3 $个球,规定谁先抓到最后一个球谁赢.如果甲先抓,那么以下推断正确的是\xx
		\twoch{若$n=4 $,则甲有必赢的策略}{若$n=6 $,则乙有必赢的策略}{若$n=9 $,则甲有必赢的策略}{若$n=11 $,则乙有必赢的策略}
				\qs 已知$ A,B $是函数$ y=2^x $的图象上的相异两点,若点$ A,B $到直线$ y=\dfrac{1}{2} $的距离相等,则点$ A,B $的横坐标之和的取值范围是\xx
		\onech{$ \left(-\infty,-1\right)$}{$ \left(-\infty,-2\right)$}{$ \left(-\infty,-3\right)$}{$ \left(-\infty,-4\right)$}
		
		
		%%%填空
		\qs 函数$f(x)=\begin{dcases}
			2^x,&x\le0\\
			x(2-x),&x>0
		\end{dcases}$的最大值为\tk;若函数$f(x)$的图象与直线$ y=k(x-1) $有且只有一个公共点,则实数$ k $的取值范围是\tk.
		\qs 若$ a=\ln\dfrac{1}{2},\ b=\left(\dfrac{1}{3}\right)^{0.8},\ c=2^{\frac{1}{3}} $,则$ a,b,c $的大小关系是\tk.
		\qs 设常数$ a\inR $,若$ \left(x^2+\dfrac{a}{x}\right)^5 $的二项展开式中$ x^7 $的系数为$ -10 $,则$ a= $\tk.
		\qs 在$\triangle ABC$中,$ H $为$ BC $上异于$ B,\ C $的任一点,$ M $为$ AH $的中点,若$ \vv{AM}=\lambda\vv{AB}+\mu\vv{AC},\  $则$ \lambda+\mu =$\tk.
		\qs 若集合$ \left\{a,b,c,d\right\} =\left\{1,2,3,4\right\}$,且下列四个关系:\\
		\ding{192}$ a=1 $\qquad \ding{193}$b\ne1$\qquad\ding{194}$c=2$,\qquad\ding{195}$d\ne4$
		有且只有一个是正确的.\par
		请写出满足上述条件的一个有序数组$ \left(a,b,c,d\right) $=\tk;符合条件的全部有序数组$ \left(a,b,c,d\right) $的个数是\tk.
		\qs 已知正方体$ABCD-A_1B_1C_1D_1$的棱长为$ 4\sqrt{2} $,点$ M $是棱$ BC $的中点,点$ P $在底面$ ABCD $内,点$ Q $在线段$ A_1C_1 $上,若$ PM=1 $,则$ PQ $的长度的最小值为\tk.
		\qs 设抛物线$ C:y^2=4x $的顶点为$ O $,经过抛物线$ C $的焦点且垂直于$x$轴的直线和抛物线$ C $交于$ A,~B $两点,则$ \abs{\vv{OA}+\vv{OB}} $=\tk.
		\qs 已知$ \left(5x-1\right)^n $展开式中,各项系数的和与各项二项式系数的和之比为$ 64:1 $,则$ n $=\tk.
		\qs 已知点$ M\left(x,y\right) $的坐标满足条件$\begin{dcases}
		x-1\le 0,\\
		x+y-1\ge 0,\\
		x-y+1\ge 0.
		\end{dcases}$设$ O $为原点,则$ \abs{OM} $的最小值是\tk.
		
		\qs 已知函数$f(x)=\begin{dcases}
		x^2-2x-3,&x>a,\\
		-x,&x\le a.
		\end{dcases}$当$ a=0 $时,$f(x)$的值域为\tk;当$f(x)$有两个不同的零点时,实数$ a $的取值范围是\tk.
		\qs 已知函数$f(x)=\begin{dcases}
		x^2+x,&-x\le x\le c,\\
		\dfrac{1}{x},&c<x\le 3.
		\end{dcases}$若$ c=0 $,则$f(x)$的值域是\tk;若$f(x)$的值域是$ \left[-\dfrac{1}{4},2\right] $,则实数$ c $的取值范围是\tk.
		\qs 对任意实数$ k $,定义集合$ D_k=\left\{\left(x,y\right)\left|\begin{dcases}
		x-y+2\ge0\\
		x+y-2\le 0\\
		kx-y\le0 	
		\end{dcases},x,y\inR
		\right.\right\} $.\par 
		\ding{192}若集合$ D_k $表示的平面区域是一个三角形,则实数$ k $的取值范围是\tk;\\
		\ding{193}当$ k=0 $时,若对任意的$ \left(x,y\right)\in D_0 $,有$ y\ge a(x+3)-1 $恒成立,且存在$ \left(x,y\right)\in D_0 $,使得$ x-y\le a $成立,则实数$ a $的取值范围是\tk.
		\clearpage
		%%%简答
		\qs 已知等差数列$\{a_n\}$的前$n$项和为$S_n$,且 $a_2=5,S_3=a_7 $.
		\begin{parts}
			\part 求数列$\{a_n\}$的通项公式;
			\part 若$ b_n=2^{a_n}, $求数列$ \left\{a_n+b_n\right\} $的前$n$项和$S_n$
		\end{parts}
\kb
\qs 已知数列$\{a_n\}$是公比为$ \dfrac{1}{3} $的等比数列,且$ a_2+6 $是$ a_1 $和$ a_3 $的等差中项.
\begin{parts}
	\part 求$\left\{a_n\right\}$的通项公式;
	\part 设数列$\{a_n\}$的前$n$项积为$T_n$,求$T_n$的最大值.
\end{parts}
	\clearpage 
		\qs 如图,在$\triangle ABC$中,$ D $为边$ BC $上一点,$ AD=6,\ BD=3,\ DC=2 $.
		\begin{parts}
		\part 若$ \angle ADB=\dfrac{\pi}{2} $,求$ \angle BAC $的大小;
		\part 若$ \angle ADB=\dfrac{2\pi}{3} $,求$ \triangle ABC $的面积.
		\end{parts}
	\mbox{\hspace{1em}}\hfill 
			\begin{tikzpicture}[scale=0.5]
		\begin{scope}
		\coordinate[label=below:\small $B$](B)at(0,0);
		\coordinate[label=below:\small$D$](D)at(3,0);
		\coordinate[label=below:\small$C$](C)at(5,0);
		\coordinate[label=left:\small$A$](A)at(3,6);
		\draw (B)--(C)--(A)--cycle;
		\draw (A)--(D);
		\node[below](c)at(2.5,-0.5) {$\text{图}1$};
		\end{scope}
		\begin{scope}[xshift=8cm]
		\coordinate[label=below:\small$B$](B)at(0,0);
		\coordinate[label=below:\small$D$](D)at(3,0);
		\coordinate[label=below:\small$C$](C)at(5,0);
		\coordinate[label=above:\small$A$](A)at(6,5);
		\draw (B)--(C)--(A)--cycle;
		\draw (A)--(D);
		\node[below](c)at(2.5,-0.5) {$\text{图}2$};
		\end{scope}
		\end{tikzpicture}
	
	\kb 
	\qs 已知函数$f(x)=\cos 2x\bm{\cdot}\tan\left(x-\dfrac{\pi}{4}\right)$.
	\begin{parts}
	\part 求函数$f(x)$的定义域;
	\part 求函数$f(x)$的值域.
	\end{parts} 
\newpage
\qs 已知函数$f(x)=2\sin ^2x-\cos\left(x+\dfrac{\pi}{3}\right)$.
\begin{parts}
	\part 求$f(x)$的最小正周期;
	\part 求证:当$ x\in\left[0,\dfrac{\pi}{2}\right] $时,$f(x)\ge -\dfrac{1}{2}$.
\end{parts}
\kongbai
\qs 如图,在$\triangle ABC$中,点$ D $在$ AC $边上,且$ AD=3DC ,\ AB=\sqrt{7},\ \angle ADB=\dfrac{\pi}{3},\ \angle C=\dfrac{\pi}{6}$.
\begin{parts}
	\part 求$ DC $的值;
	\part $ \tan \angle ABC $的值.
\end{parts}
\begin{flushright}
	\begin{tikzpicture}
		\coordinate[label=left:$A$](A) at (0,0);
		\coordinate[label=below:$D$](D) at (3,0);
		\coordinate[label=right:$C$](C) at (4,0);
		\coordinate[label=above:$B$](B) at ($(25:sqrt(7)$);
		\draw (A)--(C)--(B)--cycle;
		\draw (B)--(D);
	\end{tikzpicture}
\end{flushright}
%%%石景山整理未完成

\newpage
%%%西城文科整理完成
\qs 已知椭圆$C$:$\dfrac{x^2}{a^2}+\dfrac{y^2}{b^2}=1~(a>b>0)$过点$ A(2,0),B(0,1) $两点.
\begin{parts}
	\part 求椭圆$ C $的方程及离心率;
	\part 设点$ Q $在椭圆上,试问直线$ x+y-4=0 $上是否存在点$ P $,使得四边形$ PAQB $是平行四边形?若存在,求出点$ P $的坐标;若不存在,说明理由.
\end{parts}
\kongbai 
\qs 已知椭圆$C$:$\dfrac{x^2}{a^2}+\dfrac{y^2}{b^2}=1~(a>b>0)$过点$ A(2,0) $,且离心率为$ \dfrac{\sqrt{3}}{2} $.
\begin{parts}
	\part 求椭圆$C$的方程;
	\part 设直线$ y=kx+\sqrt{3} $与椭圆$ C $交于$ M,N $两点,若直线$ x=3 $上存在点$ P $,使得四边形$ PAMN $是平行四边形,求$ k$ 的值.
\end{parts}
\newpage
\qs 已知椭圆$C$:$\dfrac{x^2}{a^2}+\dfrac{y^2}{b^2}=1~(a>b>0)$的离心率等于$ \dfrac{1}{2} $,$ P\left(2,3\right) ,Q\left(2,-3\right)$是椭圆$C$上的两点.
\begin{parts}
	\part 求椭圆$C$的方程;
	\part $ A,B $是椭圆$C$上位于直线$ PQ $两侧的动点,当$ A,B $运动时,满足$ \angle APQ=\angle BPQ $,试问直线$ AB $的斜率是否为定值?如果为定值,请求出此定值,如果不是定值,说明理由.
\end{parts}
\kongbai
\qs 已知椭圆$ C:\dfrac{x^2}{3m} +\dfrac{y^2}{m}=1$,直线$ l:x+y-2=0 $与椭圆$C$相交于$ P,Q $两点,与$x$轴交于点$ B $,点$ P,Q $与点$ B $不重合.
\begin{parts}
	\part 求椭圆$C$的离心率;
	\part 当$ S_{\triangle OPQ} =2$时,求椭圆$C$的方程;
	\part 过原点$ O $作直线$ l $的垂线,垂足为$ N $,若$ \abs{PN}=\lambda \abs{BQ} $,求$ \lambda $的值.
\end{parts}
\newpage
\qs 已知椭圆$C$:$\dfrac{x^2}{a^2}+\dfrac{y^2}{b^2}=1~(a>b>0)$的左右焦点分别为$ F_1,\ F_2 $,点$ B\left(0,\sqrt{3}\right) $在椭圆$C$上,$ \triangle F_1BF_2 $是等边三角形.
\begin{parts}
	\part 求椭圆$C$的标准方程;
	\part 点$ A $在椭圆$C$上,线段$ AF_1 $与线段$ BF_2 $交于点$ M $,若$ \triangle MF_1F_2 $与$ \triangle AF_1F_2 $的面积之比为$ 2:3 $,求点$ M $的坐标.
\end{parts}
\kb 
\qs 已知椭圆$C$:$\dfrac{x^2}{a^2}+\dfrac{y^2}{b^2}=1~(a>b>0)$的右焦点$ F(1,0) $与短轴两个端点的连线互相垂直.
\begin{parts}
	\part 求椭圆$C$的方程;
	\part 设点$ Q $为椭圆$C$上一点,过原点$ O $且垂直于$ QF $的直线与直线$ y=2 $交于点$ P $,求$ \triangle OPQ $面积$ S $的最小值.
\end{parts}
\newpage
\qs 已知函数$f(x)=x^2\ln x-2x$.
\begin{parts}
	\part 求曲线$y=f(x)$在点$ \left(1,f(1)\right) $处的切线方程;
	\part 求证:存在唯一的$ x_0\in\left(1,2\right) $,使得曲线$y=f(x)$在点$ \left(x_0,f(x_0)\right) $处的切线的斜率为$ f\left(2\right)-f\left(1\right) $;
	\part 比较$ f(1.01) $与$ -2.01 $的大小,并加以证明.
\end{parts}
\kongbai
\qs 已知函数$f(x)=e^{ax}\cdot \sin x-1$,其中$ a>0 $.
\begin{parts}
	\part 当$ a=1 $时,求曲线$y=f(x)$在点$ \left(0,f(0)\right) $处的切线方程;
	\part 证明:$f(x)$在区间$ \left[0,\pi\right] $上恰有$ 2 $个零点.
\end{parts}
\newpage
\qs 已知函数$f(x)=\dfrac{\ln\left(x-a\right)}{x}$.
\begin{parts}
	\part 若$ a=1 $,确定函数$f(x)$的零点;
	\part 若$ a=-1 $,证明:函数$f(x)$是$ \left(0,+\infty\right) $上的减函数;
	\part 若曲线$y=f(x)$在点$ \left(1,f(1)\right) $处的切线与直线$ x-y=0 $平行,求$ a $的值.
\end{parts}
\kb
\qs 已知函数$f(x)=\left(x-1 \right)e^x+ax^2$.
\begin{parts}
	\part 求曲线$y=f(x)$在点$ \left(0,f(0)\right) $处的切线方程;
	\part 求证:“$a<0$”是“函数$f(x)$有且仅有一个零点”的充分不必要条件.
\end{parts}





	\end{questions}
\end{document}