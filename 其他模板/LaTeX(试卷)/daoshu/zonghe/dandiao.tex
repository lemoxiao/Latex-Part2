\section{函数单调性、零点问题}
\subsection{单调性}
\subsubsection{定义}
\wz{设函数$f(x)$在闭区间$ \left[a,b\right] $上连续,在开区间$ \left(a,b\right) $上可导,此时有:\par
如果在开区间$ (a,b) $内,恒有$ f'(x) >0$,则$ f(x) $在闭区间$ \left[a,b\right] $上为增函数;\par 
如果在开区间$ (a,b) $内,恒有$ f'(x) <0$,则$ f(x) $在闭区间$ \left[a,b\right] $上为减函数;\par 
如果在开区间$ (a,b) $内,恒有$ f'(x) =0$,则$ f(x) $在$ \left(a,b\right) $上为常数;\par 
}
\subsubsection{求可导函数单调性的一般方法}
\wz{
\begin{enumerate}[(1)]
\item 确定$f(x)$的定义域;
\item 求出$f'(x)$,令$f'(x)=0$,解此方程,求出它在定义域内的所有实根;
\item 把函数$f(x)$的间断点(即$f(x)$无定义的点)和上面的各实根按照由小到大的顺序排列起来,然后用这些点把$f(x)$的定义域分成若干个小区间;
\item 确定$f'(x)$在各个小区间内的符号,根据$f'(x)$的符号判定函数$f(x)$在每个相应小区间内的增减性. 
\end{enumerate} }
\subsubsection{含参问题的讨论}
\wz{当研究含有参变量的函数$f(x)$的单调性时,要对参变量进行分类讨论,常见讨论点为:
\begin{enumerate}[(1)]
\item 求导后,令$f'(x)=0$,解方程,对方程类型进行讨论;
\item 解方程的过程中,对$f'(x)$中符号确定部分忽略,可构造新函数进行化简;
\item 解方程时常分解因式,不好分解因式时采用求根公式;
\item 对根的大小进行讨论;对根是否在定义域内进行讨论;
\item 讨论过程中,一定要注意参变量的取值对导数图象是否有影响(比如开口方向).
\end{enumerate}}
\subsection{零点问题}
\wz{
\begin{description}
\item[方法一]:\begin{enumerate}[i)]
\item 求函数$f(x)$的单调区间和极值;
\item 根据函数$f(x)$的性质作出其图像;
\item 判断函数$f(x)$的零点的个数.
\end{enumerate}
\item[方法二]:\begin{enumerate}[i)]
\item 求函数$f(x)$的单调区间和极值;
\item 分类讨论,判断函数的零点.
\end{enumerate}
注意:\begin{enumerate}[(1)]
\item 研究零点时,首先要确定有没有零点,如果有,再研究有几个;
\item 研究零点个数时,对于函数自变量趋向无穷时函数值的描述,一般采用选取某个特殊的函数值来说明符合正负的方法.
\end{enumerate}
\end{description}}
\subsection{练习}
\begin{questions}
\qs 函数$f(x)=x^3-3x^2+1$是减函数的区间为\xx
\onech{$ \left(2,+\infty\right) $}{$ \left(-\infty,2\right) $}{$ \left(-\infty,0\right) $}{$ \left(0,2\right) $}
\qs 函数$ y=x\cos x-\sin x $在下面哪个区间内是增函数\xx
\onech{$\left( \dfrac{\pi}{2},\dfrac{3\pi}{2}\right) $}{$\left( \pi,2\pi\right) $}{$\left( \dfrac{3\pi}{2},\dfrac{5\pi}{2}\right) $}{$\left( 2\pi,3\pi\right) $}
\qs 若$f(x)=-\dfrac{1}{2}x^2+b\ln (x+2)$在$(-1,+\infty)$上是减函数,则$ b $的取值范围是\xx
\onech{$ \left[-1,+\infty\right) $}{$ \left(-1,+\infty\right) $}{$ \left(-\infty,-1\right] $}{$ \left(-\infty,-1\right) $}
\qs 设$f(x),~g(x)$分别是定义域在$ R $上的奇函数和偶函数,当$ x<0 $时,$ f'(x)g(x)+f(x)g'(x) >0$,~且$ g(-3)=0 $,~则不等式$ f(x)g(x)<0 $的解集是\xx
\twoch{$ \left(-3,0\right)\cup \left(3,+\infty\right) $}{$ \left(-3,0\right)\cup \left(0,3\right) $}{$ \left(-\infty,-3\right)\cup \left(3,+\infty\right) $}{$ \left(-\infty,-3\right)\cup \left(0,3\right) $}
\question
设函数$f'(x)$是奇函数$f(x)~(x\in \mathbf{R})$的导函数,$f(-1)=0$,当$x>0$时,$xf'(x)-f(x)<0$,则使得$f(x)>0$成立的$x$的取值范围是\xx
\twoch{$(-\infty,-1)\cup(0,1)$}{$(-1,0)\cup(1,+\infty)$}{$(-\infty,-1)\cup(-1,0)$}{$(0,1)\cup(1,+\infty)$}

\question
设$f(x)$,~$g(x)$是$\mathbf{R}$上的可导函数,$f'(x)$,~$g'(x)$分别是$f(x)$,~$g(x)$的导函数,且$f'(x)g(x)-f(x)g'(x)<0$,$g(x)>0$对$x\in\mathbf{R}$恒成立,则当$a<b$时,有\xx

\twoch{$f(a)g(b)>f(b)g(a)$}{$f(b)g(b)<f(a)g(a)$}{$f(b)g(b)>f(a)g(a)$}{$f(a)g(b)<f(b)g(a)$}
\question
若函数$f(x)=x-\dfrac{1}{3}\sin 2x+a\sin x$在$(-\infty,+\infty)$单调递增,则$a$的取值范围是\xx
\onech{$\left[-1,1\right]$}{$\left[-1,\dfrac{1}{3}\right]$}{$\left[-\dfrac{1}{3},\dfrac{1}{3}\right]$}{$\left[-1,-\dfrac{1}{3}\right]$}
\question
若$f(x)$的导数为$f'(x)$且满足$f'(x)<f(x)$,则$f(3)$与$e^3f(0)$的大小关系是\xx
\onech{$f(3)<e^3f(0)$}{$f(3)=e^3f(0)$}{$f(3)>e^3f(0)$}{不能确定}

\qs 若$ 0<x_1<x_2<1,~ $则\xx
\twoch{$ e^{x_2}-e^{x_1}>\ln x_2-\ln x_1 $}{$ e^{x_2}-e^{x_1}<\ln x_2-\ln x_1 $}{$ x_2e^{x_1}>x_1e^{x_2} $}{$ x_2e^{x_1}<x_1e^{x_2} $}
\qs 对于$ \mathbf{R} $上可导的任意函数$f(x)$,若满足$ (x-1)f'(x)\ge 0 $,则必有\xx
\fourch{$ f(0)+f(2)<2f(1) $}{$ f(0)+f(2)\le 2f(1) $}{$ f(0)+f(2)\ge 2f(1) $}{$ f(0)+f(2)>2f(1) $}

\qs 已知函数$f(x)=\dfrac{1}{3}x^3+x^2+ax-5$.
\begin{parts}
\part 若函数的单调递减区间是$ (-3,1) $,则$ a $的值是\tk;
\part 若函数在$ \left[1,+\infty\right) $上是单调增函数,则$ a $的取值范围是\tk.
\end{parts}
\kb
\qs
已知函数$f(x)=x^3+3ax^2+3x+1$.
\begin{parts}
\part 当$a=-\sqrt{2}$时,讨论$f(x)$的单调性;
\part 若$x \in \left[2,+\infty\right)$时,$f(x)\ge0$,求$a$的取值范围.
\end{parts}
\kb
\question
已知函数$f(x)=(x-2)e^x+a(x-1)^2$.
\begin{parts}
\part 讨论$f(x)$的单调性;
\part 若$f(x)$有两个零点,求$a$的取值范围.
\end{parts}
\kb
\qs (2011理)已知函数$f(x)=(x-k)^2e^\tfrac{x}{k}$.
\begin{parts}
\part[5]求$f(x)$的单调区间;
\part[8]若对于任意的$x\in(0,+\infty)$,都有$f(x)\le\dfrac{1}{e}$,求$k$的取值范围
\end{parts}
\kb
\qs (2011文)已知函数$f(x)=(x-k)e^x$.
\begin{parts}
\part[5]求$f(x)$的单调区间
\part[8]求$f(x)$在区间$\left[0,1\right] $上的最小值.
\end{parts}
\kb 
\question
(2016理)设函数$f(x)=x{{e}^{a-x}}+bx$,曲线$y=f(x)$在点$(2,f(2))$处的切线方程为$y=(e-1)x+4$.
\begin{parts}
\part[5]求a,b的值;
\part[8]求$f\left( x\right) $的单调区间.
\end{parts}
\kb 
\question
\kb 
(2010理)已知函数$f(x)=\ln\left(1+x\right)-x+\dfrac{k}{2}x^2~(k\ge0)$
\begin{parts}
\part[5]当$k=2$时,求曲线$y=f(x)$在点$(1,f(1))$处的切线方程;
\part[8]求$f(x)$的单调区间.
\end{parts}
 

\end{questions}
