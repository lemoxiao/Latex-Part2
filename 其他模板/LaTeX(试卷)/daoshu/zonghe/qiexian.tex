\section{切线方程}
\subsection{导数的几何意义} 
\wz{函数$y=f(x)$在$x_0$处的导数$ f'(x_0) $的几何意义是:曲线$y=f(x)$在点$ P(x_0,f(x_0)) $处的切线的斜率(瞬时速度就是位移$ s(t) $对时间$ t $的导数).}\par 
\subsection{求曲线切线方程的步骤:}
\subsubsection{点$ P(x_0,y_0) $在曲线上}
\wz{\begin{enumerate}[(1)]
\item 求出函数$y= f(x) $在点$ x=x_0 $的导数,即曲线$y=f(x)$在点$ P(x_0,f(x_0)) $处切线的斜率;
\item 在已知切点坐标$ P(x_0,f(x_0)) $和切线斜率的条件下,求得切线方程为$ y-y_0=f'(x_0)(x-x_0) $
\end{enumerate}
注:\ding{192} 当曲线$y=f(x)$在点$ P(x_0,f(x_0)) $处的切线平行于$y$轴时(此时导数不存在),由切线的定义可知,切线方程为$ x=x_0 $;\par
\ding{193} 当切点坐标未知时,应首先设出切点坐标,再求解.}
\subsubsection{点$ P(x_0,y_0) $不在曲线上}
\wz{\begin{enumerate}[1)]
\item 设出切点$P'\left(x_1,f\left(x_1\right)\right)$;
\item 写出过点$P'\left(x_1,f\left(x_1\right)\right)$的切线方程$ y-f\left(x_1\right)=f'\left(x_1\right)(x-x_1) $;
\item 将点$ P $的坐标$ \left(x_0,y_0\right) $代入切线方程,求出$ x_1 $;
\item 将$ x_1 $的值代入方程$y-f\left(x_1\right)=f'\left(x_1\right)(x-x_1) $,可得过点$ P(x_0,y_0) $的切线方程.
\end{enumerate}}
\subsubsection{切线方程已知}
当曲线的切线方程是已知时,常合理选择以下三个条件的表达式解题:
\wz{\begin{enumerate}[1)]
	\item 切点在切线上;
	\item 切点在曲线上;
	\item 切点横坐标处的导数等于切线的斜率.
\end{enumerate}
	
	
}
\subsection{练习}
\begin{questions}
\qs 直线$ l $是曲线$ y=\dfrac{1}{3}x^3-x^2+2x+1 $的切线,则$ l $的斜率的取值范围是\xx
\onech{$ \left(-\infty,1\right] $}{$ \left[-1,0\right] $}{$ \left[0,1\right] $}{$ \left[1,+\infty\right) $}
\qs 已知直线$ y=x+1 $与曲线$ y=\ln (x+a) $相切,则$ a $的值为\xx
\onech{1}{2}{$ -1 $}{$ -2 $}
\qs 垂直于直线$ 2x-6y+1=0 $且与曲线$ y=x^3+3x^2-5 $相切的直线方程是\xx
\twoch{$3x-y+6=0$}{$ 3x+y+6=0 $}{$ 3x-y-6=0 $}{$ 3x+y-6=0 $}
\qs 已知曲线$ S:y=3x-x^3 $及点$ P(2,2) $,则过点$ P $可向$ S $引切线,其切线的条数为\xx
\onech{0}{1}{2}{3} 
\qs 若曲线$ y=x^4 $的一条切线$ l $和直线$ x+4y-8=0 $垂直,则$ l $的方程为\xx
\twoch{$ 4x-y-3=0 $}{$ x+4y-5=0 $}{$ 4x-y+3=0 $}{$ x-4y+5=0 $}


\qs 已知曲线$y=x+\ln x$在点$(1,1)$处的切线与曲线$y=ax^2+(a+2)x+1$相切,则$a=$\tk.
\qs 已知$f(x)$为偶函数,当$x<0$时,$f(x)=\ln \left(-x\right)+3x$,则曲线$y=f(x)$在点$(1,-3)$处的切线方程是\tk.
\qs 曲线$ y=xe^x+2x+1 $在点$ (0,1) $处的切线方程为\tk.
\qs 设点$ P $是曲线$ y=\dfrac{x^3}{3}-x^2-3x-3 $上的一个动点,则以$ P $为切点的切线中,斜率取得最小值时的切线方程是\tk.

\kb
\qs 已知两曲线$ y=x^3+ax $和$ y=x^2+bx+c $都经过点$ P(1,2)$,且在点$ P $处有公切线,试求$ a,b,c $值.
\vspace{5em}
\qs 已知函数$f(x)=ax^3+3x^2-6ax-11,~g(x)=3x^2+6x+12$,直线$ m:~y=kx+9 $,又$ f'(-1)=0. $
\begin{parts}
\part 求$ a $的值;
\part 是否存在$ k $的值,使直线$ m $既是曲线$y=f(x)$的切线,又是曲线$y=g(x)$的切线?如果存在,求出$ k $的值,如果不存在,说明理由.
\end{parts}
\kb
\qs 是否存在这样的$ a $,使得$ f(x)=ax+\sin x $存在两切线互相垂直.
\kb 
\qs 已知函数$f(x)=x^3-x$.
\begin{parts}
\part 求曲线$y=f(x)$在点$ M(t,f(t)) $处的切线方程;
\part 设$ a>0 $,如果过点$ (a,b) $可作曲线$y=f(x)$的三条切线,证明:$ -a<b<f(a) .$
\end{parts}

\newpage
\qs 已知函数$f(x)=(x+1)\ln x-a(x-1)$.
\begin{parts}
\part 当$a=4$时,求曲线$y=f(x)$在$(1,f(1))$处的切线方程;
\part 若当$x\in (1,+\infty)$时,$f(x)>0$,求$a$的取值范围.
\end{parts}
\kb 
\qs 设函数$f(x)=\dfrac{1}{3}x^3-\dfrac{a}{2}x^2+1,~$其中$ a>0,~ $若过点$ (0,2) $可作曲线$ y=f(x) $的三条不同切线,求$ a $的取值范围.
\kb
\question
(2016文)设函数$f\left( x\right) =x^3+ax^2+bx+c$.
\begin{parts}
\part[3]求曲线$y=f\left(x\right)$在点$\left(0,f(0)\right)$处的切线方程;
\part[5]设$a=b=4$,若函数$y=f\left(x\right)$有三个不同零点,求$c$的取值范围;
\part[5]求证:$a^2-3b>0$是$y=f\left(x\right)$有三个不同零点的必要而不充分条件.
\end{parts}
\kb
\question
(2014文)已知函数$f(x)=2{{x}^{3}}-3x$.
\begin{parts}
\part[3]求$f(x)$在区间$[-2,1]$上的最大值;
\part[5]若过点$P(1,t)$存在3条直线与曲线$y=f(x)$相切,求$t$的取值范围;
\part[5]问过点$A(-1,2)$,~$B(2,10)$,~$C(0,2)$分别存在几条直线与曲线$y=f(x)$相切?(只需写出结论)
\end{parts}
\kb 
\question
(2013理)设$l$为曲线$C$:~$y=\dfrac{\ln x}{x}$在点(1,0)处的切线.
\begin{parts}
\part[5]求$l$的方程;
\part[8]证明:除切点$(1,0)$之外,曲线$C$在直线$l$的下方.
\end{parts}
\kb 
\question
(2013文)已知函数$f(x)=x^2+x\sin x+\cos x$.
\begin{parts}
\part[5]若曲线$y=f(x)$在点$(a,f(a))$处与直线$y=b$相切,求$a$与$b$的值;
\part[8]若曲线$y=f(x)$与直线$y=b$有两个不同的交点,求$b$的取值范围.
\end{parts}
\kb 
\question
(2012理)已知函数$f(x)=ax^2+1~(a>0)$,$g(x)=x^3+bx$.
\begin{parts}
\part[5] 若曲线$y=f(x)$与曲线$y=g(x)$在它们的交点$(1,c)$处具有公共切线,求$a$,$b$的值;
\part[8] 当$a^2=4b$时,求函数$f(x)+g(x)$的单调区间,并求其在区间$(-\infty,-1)$上的最大值.
\end{parts}
\kb
\question
(2012文)已知函数$f(x)=ax^2+1(a>0)$,$g(x)=x^3+bx$.
\begin{parts}
\part[5]若曲线$y=f(x)$与曲线$y=g(x)$在它们的交点$(1,c)$处具有公共切线,求$a$,$b$的值;
\part[8]当$a=3$,$b=-9$时,若函数$f(x)+g(x)$在区间$[k,2]$上的最大值为28,求$k$的取值范围.
\end{parts}
\kb
\question
设函数$f(x)=x^2+ax+b$,$g(x)=e^x(cx+d)$,若曲线$y=f(x)$和曲线$y=g(x)$都过点$P~(0,2)$,且在点$P$处有相同的切线$y=4x+2$.
\begin{parts}
\part 求$a,b,c,d$的值;
\part 若$x\ge-2$时,$f(x)\le kg(x)$,求$k$的取值范围.
\end{parts}

\end{questions}
