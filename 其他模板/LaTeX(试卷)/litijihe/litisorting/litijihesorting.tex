\documentclass{BHCexam}
\newcommand{\tbf}[1]{\item \textbf{#1}:\ }
\newcounter{dingyi}[section]
\renewcommand{\thedingyi}{\arabic{dingyi}}
\newenvironment{gongli}[1][]{\refstepcounter{dingyi}\noindent\textbf{公理 \thedingyi:\ #1} }{\hspace{\stretch{1}}\par  }%\rule{1ex}{1ex}
%\newtheorem{Therome}{公理}[section]
\newtheorem{Therome}{定理}[section]
\newtheorem{lemma}{推论}[section]
%控制目录深度
\setcounter{tocdepth}{2}
\begin{document}
\biaoti{立体几何}
\fubiaoti{DonQ}
\maketitle
\tableofcontents
\newpage
\section{平面的基本性质}
\begin{gongli}
如果一条直线的两点在一个平面内,那么这条直线上的所有点都在这个平面内 
\end{gongli}
\begin{gongli}
如果两个平面有一个公共点,那么它们还有其他公共点,且所有这些公共点的集合是一条过这个公共点的直线 
\end{gongli}
\begin{gongli}
经过不在同一条直线上的三点,有且只有一个平面.
\end{gongli}
\begin{lemma}
经过一条直线和直线外的一点有且只有一个平面
\end{lemma}
\begin{lemma}
经过两条相交直线有且只有一个平面
\end{lemma}
\begin{lemma}
经过两条平行直线有且只有一个平面
\end{lemma}


\begin{gongli}
平行于同一条直线的两条直线互相平行.
\end{gongli}
\begin{Therome}[等角定理]
若一个角的两边和另一个角的两边分别平行并且方向相同,则这两个角相等.
\end{Therome}

\begin{lemma}[等角定理的推论]
若两条相交直线和另两条相交直线分别平行,则这两条直线所成的锐角(或直角)相等.
\end{lemma}
\begin{Therome}[异面直线定理]
连结平面内一点与平面外一点的直线,和这个平面内不经过此点的直线是异面直线.
\end{Therome}
\begin{Therome}[异面直线垂直]
如果两条异面直线所成的角是直角,则叫两条异面直线垂直.
\end{Therome}
\section{平行}
\subsection{直线与平面平行判定}
\subsubsection{相关定理}
\begin{Therome}[判定定理]
平面外一条直线和此平面内的一条直线平行,则该直线与此平面平行$(\text{线线平行}\Rightarrow\text{线面平行})$.
\end{Therome}
\begin{Therome}[性质定理]
一条直线与一个平面平行,则过这条直线的任一平面与此平面的交线与该直线平行$(\text{线面平行}\Rightarrow\text{线线平行})$.
\end{Therome}
\subsubsection{证明线面平行的常用方法}
\begin{enumerate}[(1)]
\item 利用线面平行的定义;
\item 利用线面平行的判定定理:找到平面内与已知直线平行的直线.首先确定能否直接找到此直线,如果没有则可以根据以下方法:
\begin{enumerate}[i)]
\item \textbf{中位线法}:当题目中给出中点时,考虑用三角形中位线;
\item \textbf{平行四边形法}:无明显三角形构造时,用平行四边形法则,利用平行四边形的对边平行且想等证明;
\item \textbf{性质定理}:利用线面平行的性质定理证明.
\end{enumerate}
\end{enumerate}

\subsection{平面和平面平行}
\subsubsection{相关定理}
\begin{Therome}[判定定理]
一个平面内的两条相交直线与另一个平面平行,则这两个平面平行.
$(\text{线面平行}\Rightarrow\text{面面平行})$.
\end{Therome}
\begin{Therome}[性质定理]
如果两个平行平面同时和第三个平面相交,那么它们的交线平行.
\end{Therome}
\subsubsection{面面平行证明常用方法}
\begin{enumerate}[1)]
\item \textbf{利用定义}\quad  
说明平面和平面没有公共点(常用反证法)
\item \textbf{判定定理}\quad 在其中一个面内找到两条相交直线,证明线面平行.要证明线面平行,也就需要线线平行.最终也就是\textbf{中位线性质}或者\textbf{平行四边形性质} 
\end{enumerate}


\section{垂直}
\subsection{直线和平面垂直}
如果一条直线和一个平面相交,并且和这个平面内的任意一条直线都垂直,我们就说这条直线和这个平面互相垂直.其中直线叫做平面的\textbf{垂线},平面叫做直线的\textbf{垂面}.交点叫做\textbf{垂足}.
\subsubsection{常用定理}


\begin{Therome}[判定定理1]
如果一条直线与一个平面内的两条相交直线都垂直,则该直线与此平面垂直.
\end{Therome}
\begin{Therome}[判定定理2]
如果在两条平行直线中有一条垂直于平面,则另一条直线也垂直于这个平面.
\end{Therome}
\begin{Therome}[性质定理]
垂直于同一个平面的两条直线平行.
\end{Therome}
\begin{Therome}[三垂线定理]
在平面内的一条直线,如果它和这个平面的一条斜线的射影垂直,那么它也和这条斜线垂直.
\end{Therome}
\begin{Therome}[三垂线定理逆定理]
在平面内的一条直线,若和这个平面的一条斜线垂直,则它也和这条斜线的射影垂直.
\end{Therome}
\subsubsection{线线垂直证明方法}
\begin{enumerate}[1)]
\tbf{勾股定理}同一平面内两直线相交成直角;
\tbf{三垂线}异面直线所成的角为直角时,两条异面直线垂直;
\tbf{线面垂直}一条直线与一平面垂直,则这条直线垂直于平面内任一直线.
\end{enumerate}
\subsubsection{证明线面垂直的方法}
\begin{enumerate}[1)]
\tbf{面面垂直性质定理}两平面垂直,在一个平面内垂直于交线的直线必垂直于另一个平面;
\tbf{线面垂直的判定定理}一条直线与平面内的两条相交直线都垂直.
\end{enumerate}
\subsection{平面与平面垂直的判定和性质}
\subsubsection{相关定理}
\begin{Therome}[判定定理]
一个平面过另一个平面的一条垂线,则这两个平面相互垂直.
\end{Therome}
\begin{Therome}[性质定理]
两个平面互相垂直,则一个平面内垂直于交线的直线与另一个平面垂直.
\end{Therome}
\subsubsection{证明面面垂直的方法}
线面垂直、面面垂直最终归纳为线线垂直,共面直线垂直常用勾股定理的逆定理、等腰三角形的性质;异面直线垂直的通常利用线面垂直(三垂线定理)或者空间向量.
\section{空间向量}
\subsection{基本定理}
\begin{enumerate}[1)]
\item 共线向量:对空间中任意两个向量$\vv{a},\vv{b}(\vv{b}\ne\vv{0})$,$\vv{a}\sslash\vv{b}$的充要条件是:存在唯一的实数$\lambda$,使得$\vv{a}=\lambda\vv{b}$.
\item 共面向量定理:如果两个向量$\vv{a},\vv{b}$不共线,那么向量$\vv{p}$与向量$\vv{a}$,$\vv{b}$共面的充要条件是存在唯一的有序数对$(x,y)$,使得$\vv{p}=x\vv{a}+y\vv{b}$.
\item 空间向量基本定理:如果三个向量$\vv{a}$,$\vv{b}$,$\vv{c}$不共面,那么对空间任一向量$\vv{p}$,存在有序实数组${x,y,z}$,使得$\vv{p}=x\vv{a}+y\vv{b}+z\vv{c}$.其中${\vv{a},\vv{b},\vv{c}}$叫做空间的一组基底(空间直角坐标系就是其中的一个特例,注意坐标系需要满足右手定则)
\end{enumerate}
\subsection{空间向量基本运算}
	{\kaishu 设点$ A(x_A,y_A,z_A),\ B(x_B,y_B,z_B) $,非零向量$ \vv{a}=(x_1,y_1,z_1),\ \vv{b}=(x_2,y_2,z_2) :$}\par
\begin{enumerate}[1)]
\tbf{$\vv{AB}$表示}$\vv{AB}=(x_B-x_A,y_B-y_A,z_B-z_A)$;
\tbf{距离公式}$ d_{AB}=\abs{\vv{AB}}=\sqrt{(x_A-x_B)^2+(y_A-y_B)^2+(z_A-z_B)^2} $;
\tbf{向量数量积}$\vv{a}\bm{\cdot}\vv{b}=\left|\vv{a}\right|\left|\vv{b}\right|\cos\left<\vv{a},\vv{b}\right>=x_1x_2+y_1y_2+z_1z_2$;
\tbf{夹角公式}$\cos \left<\vv{a},\vv{b}\right>=\dfrac{\vv{a}\vv{b}}{\left|\vv{a}\right|\left|\vv{b}\right|}=\dfrac{x_1x_2+y_1y_2+z_1z_2}{\sqrt{x_1^2+y_1^2+z_1^2}\sqrt{x_2^2+y_2^2+z_2^2}}$
\tbf{垂直判定} $\vv{a}\bot\vv{b}\Leftrightarrow\vv{a}\bm{\cdot}\vv{b}=0$;
\tbf{平行判定}$\vv{b}=\lambda \vv{a}\Leftrightarrow \dfrac{x_2}{x_1}=\dfrac{y_2}{y_1}=\dfrac{z_2}{z_1}$
\end{enumerate}
\subsubsection{点线面证明问题}
\begin{enumerate}[1)]
\item 三点$ P,A,B $共线:对空间任意一点$O$,有$\vv{OP}=x\vv{OA}+(1-x)\vv{OB}$(平面中也有相同性质,称作定比分点问题,部分地方省份仍然作为高考内容)
\item 四点$M,P,A,B$共面:对空间任意一点$O$,有$\vv{OP}=x\vv{OA}+y\vv{OB}+(1-x-y)\vv{OM}$.
\end{enumerate}

\subsection{方向向量和法向量}
\begin{Therome}[直线方向向量]
$l$是空间一直线,$A,B$是直线$l$上任意两点,则称$\vv{AB}$为直线$l$的方向向量,与$\vv{AB}$平行的任意非零向量也是直线$l$的方向向量
\end{Therome}
\begin{Therome}[平面法向量]
与一个给定平面垂直的向量,称作此平面的一个法向量.
\end{Therome}

%2017.12.21更新,待补充
设$ \vv{a},\ \vv{b} $是给定平面$ \alpha $内两不共线向量,$ \vv{n} $是平面$ \alpha $的法向量,则求法向量的方程组为$\begin{dcases}
\vv{n}\bm{\cdot}\vv{a}=0\\
\vv{n}\bm{\cdot}\vv{b}=0
\end{dcases}$
{\kaishu \textbf{注:}高中部分法向量主要目的为计算夹角,法向量长度并不会影响角度计算,故而在解上述方程时,可以令某一非零坐标分量为$ 1 $(或其他非零数值),进而求得另外两个坐标分量的数值.}

\subsection{平行和垂直的证明}
\subsubsection{平行的证明}
\begin{enumerate}[1)]
\item 设直线$l_1$和$l_2$的方向向量分别为$\vv{v_1}$和$\vv{v_2}$,则$l_1\sslash l_2$(或$l_1$与$l_2$重合)$\Leftrightarrow\vv{v_1}\sslash \vv{v_2}$;
\item 设直线$ l $的方向向量为$ \vv{v} $,与平面$ \alpha $共面的两个不共线向量$ \vv{v_1},\ \vv{v_2} $,$ l\sslash\alpha $或$ \l\subset\alpha \Leftrightarrow$存在两个实数$ x,\ y $使得$ \vv{v}=x\vv{v_1}+y\vv{v_2} $;
\item 设直线$ l $的方向向量为$ \vv{v} $,平面$ \alpha $的法向量为$ \vv{u} $,则$ l\sslash\alpha $或$ l\subset\alpha\Leftrightarrow\vv{v}\bot\vv{u} $;
\item 设平面$ \alpha $和$ \beta $的法向量分别为$ \vv{u_1},\vv{u_2} $,则$ \alpha\sslash\beta\Leftrightarrow\vv{u_1}\sslash\vv{u_2} $
\end{enumerate}
\subsubsection{垂直的证明}
\begin{enumerate}
	\item 设直线$ l_1 $和$ l_2 $的方向向量分别为$ \vv{v_1},\ \vv{v_2} $,则$ l_1\bot l_2 \Leftrightarrow\vv{v_1}\bot\vv{v_2}\Leftrightarrow\vv{v_1}\bm{\cdot}\vv{v_2}=0$;
	\item 设直线$ l $的方向向量为$ \vv{v} $,平面$ \alpha $的法向量为$ \vv{u} $,则$ l\bot\alpha\Leftrightarrow\vv{v}\sslash\vv{u} $;
	\item 设平面$ \alpha\text{和}\beta $的法向量分别为$ \vv{u_1}\text{和}\vv{u_2} $,则$ \alpha\bot\beta\Leftrightarrow\vv{u_1}\bot\vv{u_2} \Leftrightarrow\vv{u_1}\bm{\cdot}\vv{u_2}=0$.
\end{enumerate}
\begin{center}
	%指定列宽度,指定居中
	\begin{tabular}{>{\columncolor[rgb]{.8,.9,.9}}c|p{8cm}<{\centering}}
		\hline 
		\textbf{线线平行}&$l\sslash m\Leftrightarrow \bm{a}\sslash\bm{b}\Leftrightarrow\bm{a}=k\bm{b}\left(k\inR\right)$\\
		\hline
		\textbf{线面平行}&$l\sslash\alpha\Leftrightarrow\bm{a}\bot\bm{u}\Leftrightarrow\bm{a\cdot u}=0 $\\
		\hline
		\textbf{面面平行}&$\alpha\sslash\beta\Leftrightarrow\bm{u\sslash v}\Leftrightarrow \bm{u}=k\bm{v}$\\
		\hline  
		\textbf{线线垂直}&$l\bot m\Leftrightarrow \bm{a\bot b}\Leftrightarrow \bm{a\cdot b }=0$\\
		\hline
		\textbf{线面垂直}&$l\bot \alpha\Leftrightarrow \bm{a}\sslash\bm{u}\Leftrightarrow\bm{a}=k\bm{u}\left(k\inR\right)$\\
		\hline 
		\textbf{面面垂直}&$\alpha\bot \beta\Leftrightarrow\bm{u\bot v}\Leftrightarrow\bm{u\cdot v}=0$\\
		\hline
	\end{tabular}

\end{center}
\subsection{夹角问题}
\subsubsection{异面直线夹角}
设$ \vv{a},\ \vv{b} $分别是两异面直线$ l_1,l_2 $的方向向量,设$ l_1\text{和}l_2 $的夹角为$ \theta \left(\theta\in\left(0,\dfrac{\pi}{2}\right]\right)$,$ \vv{a}\text{和}\vv{b} $的夹角为$ \beta\left(\beta\in\left[0,\pi\right]\right) $,则有\[\cos\theta=\left|\cos\beta\right|=\left|\dfrac{\vv{a}\bm{\cdot}\vv{b}}{\abs{\vv{a}}\abs{\vv{b}}}\right|\]
\subsubsection{直线与平面所成角}
设直线$l$的方向向量为$\vv{a}$,平面$\alpha$的法向量为$\vv{n}$,直线$l$与平面$\alpha$所成的角为$ \theta $
,$\vv{a}$与$\vv{n}$的夹角为$\beta$,则\[\sin\theta=\left|\cos\beta\right|=\left|\dfrac{\vv{a}\bm{\cdot}\vv{n}}{\left|\vv{a}\right|\left|\vv{n}\right|} \right| \]
\subsubsection{二面角的大小}
\begin{enumerate}
	\tbf{几何法}设$ AB,\ CD $分别是二面角$ \alpha-l-\beta$的两个面内与棱$ l $垂直的直线,则二面角的大小$ \theta=\left< \vv{AB},\vv{CD}\right> $
	\tbf{向量法}设$ \vv{n_1},\vv{n_2} $分别是二面角$ \alpha-l-\beta $的两个半平面$ \alpha,\ \beta $的法向量,则二面角的大小$ \theta $满足$ \abs{\cos\theta}=\abs{\cos\left< \vv{n_1},\vv{n_2}\right>} $,二面角的平面角大小是向量$ \vv{n_1} $与$ \vv{n_2} $的夹角(或其补角,视觉判断是最快而方便的).
	\subsubsection{点到平面的距离}
已知点$ A $在平面$ \alpha $上,$ AB $为平面$ \alpha $的一条斜线段,$ \vv{n} $为平面$ \alpha $的法向量,则$ B $到平面$ \alpha $的距离为\[ d=\dfrac{\abs{\vv{AB}\bm{\cdot}\vv{n}}}{\abs{\vv{n}}} \]	
\end{enumerate}




\end{document}