\documentclass[marginline,noindent,answers,adobefonts]{BHCexam}	


\newcommand{\R}[1]{\in \mathbf{#1}}
\newcommand{\kj}[1]{\begin{center}\begin{tikzpicture}#1\end{tikzpicture}\end{center}}
\begin{document}
\biaoti{函数综合}
\fubiaoti{}
\maketitle
\begin{questions}
\question
已知函数$f(x)~(x\in \mathbf{R})$满足$f(-x)=2-f(x)$,若函数$y=\dfrac{x+1}{x}$与$y=f(x)$图象的交点为$(x_1,y_1),(x_2,y_2),\cdots,(x_m,y_m)$,则$\sum\limits_{i=1}^{m}(x_i+y_i)=$\xx
\onech{$0$}{$m$}{$2m$}{$4m$}

\question
设函数$f(x),g(x)$的定义域都为$\mathbf{R}$,且$f(x)$是奇函数,$g(x)$是偶函数,则下列结论正确的是\xx
\fourch{$f(x)g(x)$是偶函数}{$\abs{f(x)}g(x)$是奇函数}{$f(x)\abs{g(x)}$是奇函数}{$\abs{f(x)g(x)}$是奇函数}
\question
已知函数$f(x)$的定义域为$(-1,0)$,则函数$f(2x+1)$的定义域为\xx
\onech{$(-1,1)$}{$(-1,-\dfrac{1}{2})$}{$(-1,0)$}{$(\dfrac{1}{2},1)$}

\qs 下列函数中,其定义域和值域分别与函数$y=10^{\lg x}$的定义域和值域相同的是\xx
\onech{$y=x$}{$y=\lg x$}{$y=2^x$}{$y=\dfrac{1}{\sqrt{x}}$}
\qs
设函数$\fx$的图象与$y=2^{x+a}$的图象关于直线$y=-x$对称,且$f(-2)+f(-4)=1$,则$a=$\mbox{\hspace{1pt}}\hfill\xx
\onech{$-1$}{$1$}{$2$}{$4$}
\qs 设函数$\fx=\ln (1+\abs{x})-\dfrac{1}{1+x^2}$,则使得$\fx>f(2x-1)$成立的$x$的取值范围是\xx
\twoch{$\left(\dfrac{1}{3},1\right)$}{$\left(-\infty,\dfrac{1}{3}\right)\cup(1,+\infty)$}{$\left(-\dfrac{1}{3},\dfrac{1}{3}\right)$}{$\left(-\infty,\dfrac{1}{3}\right)\cup\left(\dfrac{1}{3},+\infty\right)$}
\qs 已知函数$f(x)=x^3-6x^2+9x-abc,~a<b<c,~$且$ f(a)=f(b)=f(c) ,~$给出如下结论:\\
\ding{192} $ f(0)f(1)>0 ;$\quad \ding{193}$f(0)f(1)<0;$\quad \ding{194}$f(0)f(3)>0;$\quad \ding{195}$f(0)f(3)<0;$\\
其中正确的结论的序号是\xx
\onech{\ding{192}\ding{194}}{\ding{192}\ding{195}}{\ding{193}\ding{194}}{\ding{193}\ding{195}}
\qs 已知函数$f(x)=\ln \left(\sqrt{1+9x^2}-3x\right)+1$,则$ f(\lg2)+f\left(\lg\dfrac{1}{2}\right) $等于\xx
\onech{-1}{0}{1}{2}
\qs 若$ a>b>1,0<c<1 $,则\xx
\twoch{$ a^c<b^c $}{$ab^c<ba^c$}{$a\log_bc<b\log_ac$}{$ \log_ac<\log_bc $}
\qs 已知$ x,y\in \mathbf{R} $,且$ x>y>0 $,则\xx
\fourch{$ \dfrac{1}{x}-\dfrac{1}{y}>0 $}{$\sin x-\sin y>0$}{$\left(\dfrac{1}{2}\right)^x-\left(\dfrac{1}{2}\right)^y<0$}{$\ln x+\ln y>0$}
\qs 如果,函数$f(x)=$的图象为折线$ ACB $,则不等式$ f(x)\ge \log_2(x+1) $的解集是\xx

\kj{\draw[->] (-2,0)--(3,0) node[below]{$x$};
\draw[->] (0,-1)--(0,2.8) node[left]{$y$};
\draw (-1,0)--(0,2)--(2,0);
\node [above left](A) at (-1,0){$A$} ;
\node [above right](B) at (2,0){$B$} ;
\node [above right](C) at (0,2){$C$} ;
\node [below left](O) at (0,0){$O$} ;
\node [below ](A1) at (-1,0){$-1$} ;
\node [below](B1) at (2,0){$2$} ;
\node [above left](C1) at (0,2){$2$} ;}
\vspace{-2em}
\twoch{$ \{x|-1<x\le0\} $}{$ \{x|-1\le x\le 1\} $}{$ \{x|0 \le x\le1\} $}{$ \{x|-1 \le x\le2\} $}
\qs  下来函数中,定义域为$ \mathbf{R} $且为增函数的是\xx
\onech{$y=e^{-x}$}{$ y=x $}{$y=\ln x$}{$y=\left|x\right|$}
\qs 函数$f(x)$的图象向右平移一个单位长度,所得图象与$ y=e^x $关于$ y $轴对称,则$\fx=$\\\mbox{\hspace{1ex}}\hfill\xx
\onech{$ e^{x+1} $}{$e^{x-1}$}{$ e^{-x+1} $}{$ e^{-x-1} $}
\qs 已知函数$f(x)$是定义在$ \mathbf{R} $上的偶函数,且在区间$ \left[0,+\infty\right) $上单调递增,若实数$ a $满足$ f(log_2a) +f(\log_\frac{1}{2}a)\le 2f(1)$,则$ a $的取值范围是\xx
\twoch{$ \left[1,2\right]$}{$ \left(0,\dfrac{1}{2}\right]$}{$ \left[\dfrac{1}{2},2\right]$}{$ \left(0,2\right]$}


\qs 已知$ \log_{18}9=a~(a\ne 2),~18^b=5. ~$求$ \log_{36}45= $\tk.
\question
若函数$f(x)=x\ln\left(x+\sqrt{a+x^2}\right)$为偶函数,则$a$=\tk.
\qs
设函数$f(x)=\Bigg\{\begin{aligned}
&~2^x-a,& x<1;\\
&~4(x-a)(x-2a),& x\ge 1.
\end{aligned}$\\
\ding{192}~若$a=1$,则$ \fx $的最小值为\tk;\\
\ding{193}~若$ \fx $恰有$ 2 $个零点,则实数$ a $的取值范围是\tk.
\qs 设函数$f(x)=\begin{dcases}
x^3-3x,&x\le a \\
-2x,&x>a.
\end{dcases}$\\
\ding{192} 若$ a=0 $,则$f(x)$的最大值为\tk;\\
\ding{193} 若$f(x)$无最大值,则实数$ a $的取值范围是\tk.
\question
若函数$f(x)=(1-x^2)(x^2+ax+b)$的图象关于直线$x=-2$对称,则$f(x)$的最大值是\tk.
\qs 已知函数$f(x)=\Bigg\{\begin{aligned}
&\dfrac{2}{x},&x\ge2\\
&(x-1)^3,&x<2.
\end{aligned}$若关于$ x $的方程$ f(x)=k $有两个不同的实根,则数$ k $的取值范围是\tk.



\qs 已知函数$f(x)=\dfrac{x^2}{1+x^2},~$那么$ f(1)+f(2)+f(\dfrac{1}{2})+f(3)+f(\dfrac{1}{3}+f(4)+f(\dfrac{1}{4})= $\tk.

\qs 已知函数$f(x)=m(x-2m)(x+m+3),~g(x)=2^x-2.$若同时满足条件:\\
\ding{192} $ \forall x \in \mathbf{R} ,~f(x)<0~\text{或}~g(x)<0;$\\
\ding{193} $ \exists \in (-\infty,-4),~f(x)g(x)<0, $\\
则$ m $的取值范围是\tk.
\qs 曲线$C$是平面内与两个定点$ F_(-1,0) $和$F_2(1,0)$的距离的积等于常数$ a^2 $的点的轨迹.给出下列三个结论:\\
\ding{192} 曲线$ C $过坐标原点;\\
\ding{193} 曲线$ C $关于坐标原点对称;\\
\ding{194} 若点$ P $在曲线$ C $上,则$ \triangle F_1PF_2 $的面积大于$ \dfrac{1}{2}a^2. $\\
其中,所有正确的结论的序号是\tk.


\end{questions}
\end{document}