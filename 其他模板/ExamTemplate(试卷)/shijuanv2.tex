\documentclass[11pt,twoside]{ctexart}
\usepackage{geometry}
\usepackage{examanswersheetv2,caption2,pifont,ulem}
\mifengxian
\begin{document}
%电脑上需要安装相应字体,分别是方正小标宋和楷体
%
%	试卷标题
%
\begin{center}\bs{}

\zihao{-2}{安阳师范学院~~~\underline{15级旅游管理专升本、本科}~~~专业~~~\underline{经济数学}~~~课}\\
{\kaishu\zihao{4}{$2015$——$2016$学年度第一学期期末考试试卷(A卷)}}
\end{center}

\zongfenlana

%
%	试卷正面
%
\begin{enumerate}
\item[\kaishu{}一]{\makebox[2mm][r]{、}\kaishu{}\dafenlan{}选择题(每小题3分,共24分)}%-------------------------------------选择题
\item 计算$|-1|=$
\xxs{1}{-1}{0}{2}

% \item 用平面去截一个几何体,如截面为长方形,则几何体不可能是
% \xx{圆柱}{圆锥}{长方体}{正方体}
\item 下列算式正确的是
\xxs{-3^2=9}{(-\dfrac{1}{4})\div(-4)=1}{(-8)^2=-16}{-5-(-2)=-3}

\item 如图(1)是小正方体图(2)的表面展开图,小正方体的六个面中和“构”字相对的面上的字为
\xx{和}{谐}{社}{会}

% \makebox[6.5cm][c]{\includegraphics[scale=1.8]{t5.png}}
% \makebox[6.5cm][c]{\raisebox{24pt}{\includegraphics[scale=2]{t7.png}}}

\item 若$a~,~b~,~c$三个数在数轴上的位置如图所示,则下列式子正确的是
\xxs{b+c>0}{c<2c}{c+a>0}{b-a>0}


\item 下列式子中:$x^2+2~,~\dfrac{1}{a}+4~,~\dfrac{3ab^2}{7}~,~\dfrac{ab}{\pi}~,~0~,~-5x$~,~单项式的个数是
\xx{6个}{5个}{4个}{3个}
\item 下列说法正确的是
\xx
{单项式$y$的次数是1,系数是0}
{多项式$\dfrac{3(1-x^2)}{8}$中$x^2$的系数是$-\dfrac{3}{8}$}
{多项式$t-5$的项是$t$和5}
{$\dfrac{xy-1}{2}$是单项式}
\item 下列计算正确的是
\xxs
{6a-5a=1}
{-(a-b)=-a+b}
{a+2a^2=3a^2}
{2(a+b)=2a+b}
\item 下列各组整式中,不属于同类项的是
\xx
{$-3a$与$a$}
{$-100$与$0.5$}
{$-2x^2y^3$与$-\dfrac{2y^3x^2}{3}$}
{$\dfrac{1}{2}a^2b$与$-3b^2a$}

\item[\kaishu{}二]{\makebox[2mm][r]{、}\kaishu{}\dafenlan{}填空题(每空3分,共21分)}%--------------------------------------填空题

\item $-\dfrac{5}{3}$的倒数是\line{2}.
\item 单项式$-\dfrac{2x^2y}{3}$的次数是\line{2}.
\item 2016年双11天猫销售额1207亿~.~用科学记数法表示1207亿元为\line{2.5}元.

\item 绝对值大于1而小于3的整数和为\line{2}.

\item 某商品原价$p$元,先提价10\%,再打九折以后的出售价格是\line{2}元(写出化简后的结果).
\item 若$|x+3|+(y-2)^2=0$,则$(x+y)^{2016}=\line{2}$.
\item 若$3x^ny^{2m+1}$与$-\dfrac{1}{2}xy^3$是同类项,则$m+n=\line{2}$~.
% \item 若$x$是2的相反数,$|y|=3$,则$x-y=$\line{2}~.


\item[\kaishu{}三]{\makebox[2mm][r]{、}\kaishu{}\dafenlan{}解答题(共55分)}%-----------------------------------------------解答题
\item {\kaishu{}计算(每题4分)}\par
\makebox[7.5cm][l]{(1)~$-17+(-16)-(-21)-13$}
\makebox[7.5cm][l]{(2)~$-5m^2n-2mn+6nm^2-3mn$}
\vfill
\makebox[7.5cm][l]{(3)~$4y^2-[3y-(3-2y)+2y^2]$}
\makebox[7.5cm][l]{(4)~$\left(\dfrac{2}{3}-\dfrac{1}{12}-\dfrac{4}{15}\right)\times(-60)$}
% \vfill
% \makebox[7.5cm][l]{\ding{176}~~$-2^2\div\dfrac{2}{3}-(-\dfrac{2}{3})\times(-30)$}
\vfill
\newpage
%
%	试卷背面
%
\item (6分)先化简,再求值~.~其中$a=1$~.\\
\makebox[10cm][c]{$\dfrac{1}{4}(-4a^2+2a-8)-2(\dfrac{1}{4}a-1)-1$}
% \makebox[10cm][c]{$-2x^2+y^2-(2y^2-3x^2)+2(y^2-2x^2)$}
\vfill
% \item (7分)某城市出租车收费标准如下:3公里以内(含3公里)收费8元,超过3公里的部分每公里收费1.5元.\\
% (1)若行驶$x$公里($x$为整数),试用含$x$的代数式表示应收的车费;\\
% (2)若某人乘坐出租汽车行驶8公里,则应付车费多少元?

\item (8分)“十一”黄金周期间,某市风景区在7天假期中每天旅游的人数变化如下表(\uwave{正数表示比前一天多的人数,负数表示比前一天少的人数}):\par\hfill\null
\begin{tabular}{|c|*{7}{|p{0.88cm}<{\centering}}|}
\hline
日期&1日&2日&3日&4日&5日&6日&7日\\\hline
人数变化(单位:万人)&1.6&0.8&0.4&$-0.4$&$-0.8$&0.2&$-1.2$\\\hline
\end{tabular}\hfill\null\par
已知9月30日的游客人数为2万人,请回答下列问题:\\
(1)七天内游客人数最多的是哪天,最少的是哪天?它们相差多少万人?\\
%3.6,4.4,4.8,4.4,3.6,3.8,2.6
(2)求这7天的游客总人数是多少万人.
%27.2
\vfill
\item (7分)如图,在一个长方形休闲广场的四角都设计一块半径相同的四分之一圆形的花坛.若广场的长为$a$米,宽为$b$米,圆形的半径为$r$米~.~\\
(1)~~请列式表示广场空地的面积;\\
(2)~~若休闲广场的长为100米,宽为50米,圆形花坛的半径为10米,\\
{\color{white}(1)}~~求广场空地的面积~.~(\uwave{计算结果保留$\pi$})\\
% \makebox[14cm][r]{\includegraphics[scale=0.55]{t23.png}}
\vfill
\newpage
\item (4分)下图是由一些火柴棒搭成的图案,请观察图案并填表。\par\hfill\null
\begin{tabular}{|c|*{6}{|p{1.2cm}<{\centering}}|}
\hline
五边形的个数&1&2&3&4&$\cdots$&$n$\\\hline
火柴棒的根数&5&9~&~&~&$\cdots$&\\\hline
\end{tabular}\hfill\null

% \hfill\null\makebox[13cm][c]{\includegraphics[scale=2]{t19.png}}\hfill\null

% \vfill
\item (6分)一种蔬菜$x$千克,不加工直接出售每千克可卖$y$元;如果经过加工重量减少了20\%,价格增加了40\%~.~问:
$x$千克这种蔬菜加工后可卖多少钱?
\vfill
\item (8分)在一次抗震救灾中,某市组织20辆汽车装运食品、药品、生活用品三种救灾物资到灾民安置区,按计划每辆汽车只能装运一种救灾物资且必须装满~.~已知用了$a$辆汽车装运食品,用了$b$辆汽车装运药品,其余剩下的汽车装运生活用品,根据表中提供的信息,解答下列问题:\par\hfill
\begin{tabular}{|c|*{3}{|p{2cm}<{\centering}}|}
\hline
物资种类&食品&药品&生活用品\\\hline
每辆汽车运载(吨)&6&5&4\\\hline
每吨所需运费(元)&120&160&100\\\hline
\end{tabular}\hfill\null\par
(1)20辆汽车共装载了多少吨救灾物资?\par
(2)装运这批救灾物资的总费用是多少元?
\vfill
% \newpage
\end{enumerate}
\end{document}