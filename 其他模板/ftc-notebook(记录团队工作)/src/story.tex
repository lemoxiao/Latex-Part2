\begin{TeamStory}{Resilience through Innovation \& Simplicity}

We are a fourth year 4-H team with 3 new members. We are a diverse
group of 11 boys and girls, in grades 8 to 12 from 6 different school
districts, and while we may speak up to 6 different languages, we are
united by a common passion for STEM.\\ \vspace{3mm}

\RawPict{src/images/robocracy2018.jpg}{.6}{} \\ \vspace{3mm}

Our theme this year is resilience through innovation and
simplicity. Last year, after finishing first place at our Hudson
Valley Regionals, we came back dead last from Eastern
SuperRegionals. We took this opportunity to take a hard look at our
process. After last year’s season, where we strove for innovation for
the sake of having a cool cutting-edge design, we have learned from
our mistakes and are now striving for resilience through innovation
and simplicity.

For the first time, we kick-started our season with the “build a robot
in 36 hours” challenge. During this time, we were able to efficiently
flush out a design and develop an intuition for the game. Because of
this, we were able to break down our process and make sure to have
deliberate design decisions that focus on resilience and efficient
simplicity, in addition to innovation. Our process emphasizes analysis
of competing ideas, developed by competing design groups, which are
considered head to head until the stronger idea wins.

We maintain our process for sustainability which we cultivated last
year, to become a more efficient and sustainable team.  We each strive
to learn two new skills during the season and we have a strong culture
of mentoring each other. This protects the team from losing skills
when someone graduates from the team.

An important part of being a member of Team Robocracy is making time
for our robust outreach in the community. We seek to empower other
kids to develop skills that they can use for the rest of their lives,
thereby building their own resilience. We also share our expertise and
skills where we can have a positive impact in the lives of others. We
run multiple afterschool enrichment programs targeting underserved
communities, run robotics camps, 3D print prosthetics, and recycle
computers to donate to third world schools.

We are very grateful for our membership in 4H. 4H provides for us an
excellent platform for our outreach and has enabled us to reach many
communities that would otherwise not be exposed to STEM and
robotics. 4H also gives us access to important resources such as
advertisement, Lego Mindstorm kits, and their liability insurance for
our workshops! Our Off The Streets and Amazing Afternoons programs in
Mt. Vernon Elementary Schools are both conducted through 4H. As the
only STEM-based 4H club in our area, we also take seriously our role
of promoting and inspiring interest in STEM at fairs and all of the
outreach we do.

\end{TeamStory}
