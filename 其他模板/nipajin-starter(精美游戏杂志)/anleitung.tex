%========================================================================
% NIP'AJIN Autorenpaket v1.1, (C) Markus Leupold-Löwenthal
%========================================================================
% Dieses Werk untersteht folgender Lizenz:
% Namensnennung–Weitergabe unter gleichen Bedingungen 3.0 Österreich
% (CC BY-SA 3.0) http://creativecommons.org/licenses/by-sa/3.0/at/
%========================================================================

\banner{Anleitung zum Autorenpaket}{Anleitung}\zlabel{Anleitung}

\begin{center}\itshape
Dieser Abschnitt gibt ein paar Hilfestellungen, wie das \nipajin-Autorenpaket verwendet werden kann. Natürlich sollte dieser Abschnitt im finalen Dokument vom Autor entfernt werden ;)
\end{center}

\begin{multicols}{2}

\mysection{Einführung}

Hallo! Schön, dass du dieses Autorenpaket verwenden möchtest. Es ermöglicht dir, mit relativ wenig Aufwand ein komplettes Rollenspiel zu schreiben und bietet dir als Ausgangspunkt eine vollständige Dokumentstruktur inkl. Cover, Layout und Regelanhang, damit eigenständige Werke entstehen können, die du auch veröffentlichen darfst. Die folgenden Punkte gilt es dabei zu beachten:

\aufzaehlung{
\item Technische Details, wie man das Autorenpaket bedient, findest du in der README Datei.
\item Der hier enthaltene Dokumentaufbau ist nur als Vorschlag zu betrachten.
\item Beachte, dass das \nipajin~Logo nicht frei ist, d.\,h. dass du nur die textuelle Version \zitat{\nipajin} verwenden darfst. Ebenso ist das \ludusleonis-Logo (Münze), sowie das TRiAS Logo nicht frei, bitte verzichte auf eine Verwendung und beachte die korrekte Groß-/Kleinschreibung dieser Namen.
\item Die im Autorenpaket enthaltenen Bilder (Cover, Seitenhintergrund, Überschriftenbanner) sind frei, d.h. du darfst sie 1:1 verwenden, oder abwandeln.
\item Dieses Dokument unterliegt der CC BY-SA. Bitte informiere dich auf der Creative Commons Webseite genau, was es bedeutet, auf so einem Werk aufzubauen. Du musst nämlich diese Lizenz auf \emph{alles}, was du davon ableitest, wieder anwenden. Das gilt auch, wenn du nur Teile in andere Werke übernimmst, \zB~dir \zitat{nur das Layout abschaust}. Weiters musst du dein gesamtes entstehendes Werk unter die CC BY-SA stellen -- du kannst also nicht Teile (\zB~Bilder!) davon aussnehmen. 
\item Bedenke auch, dass die CC BY-SA die kommerzielle Nutzung deiner Inhalte durch andere erlaubt und du daher auch selbst die nötigen Rechte haben musst, alles, was du integrierst (\zB~Bilder oder Texte aus anderen Quellen) unter diese Lizenz stellen zu dürfen.
\item Die CC BY-SA sagt \uA~aus, dass du dich zur Ableitung vom Original bekennen musst. Am einfachsten ist das, wenn du den Kasten im Impresssum \zitat{Dieses Werk nutzt das freie NIP’AJIN-System~\ldots} drinnen lässt. Damit bin ich \zitat{in der von mir festgelegten Weise} genannt und es ist leicht ersichtlich, wo das Original her kommt.
\item Wenn du Änderungen am Regelwerk vornimmst, darfst du das natürlich tun, mach aber bitte deutlich, dass und wo du dich vom Original entfernst. Verwende in diesem Fall nicht mehr \zitat{NIP'AJIN} als Namen, sondern gib der Abwandlung einen neuen Namen. Alternativ und für den Leser einfacher ist vermutlich, wenn du Settingregeln im Szenarioteil anführst und den Anhang intakt lässt. 
}

\noindent\textbf{Du bist selbst dafür verantwortlich, dass dein abgeleitetes Werk mit der CC BY-SA verträglich ist!}

\mysection{Offene Punkte}

Dies ist eine erste Version des Autorenpakets, daher ist natürlich noch Luft nach oben. Noch nicht enthalten, aber (für irgendwann) geplant, sind \uA:

\aufzaehlung{
\item Symbole: Die RPGDings Schrift, die ich für \nipajin~und \trias~entworfen habe, muss ich erst gesondert unter eine offene Lizenz stellen.
\item NSC-Blöcke: Die erstmals in Kurai Jikan eingeführten, runden NSC-Statistik-Blöcke mit runden Ecken.
\item CMYK/Drucksupport: z.\,Z. produziert das Autorenpaket RGB-PDFs, die sich für Digitaldruckereien aber nicht für den Offsetdruck eignen. Da ich hier ein komplexeres Script-Geflecht zur Konvertierung habe, ist das nicht so leicht, das für Außenstehende leicht verständlich zu extrahieren. 
\item Portierung des Autorenpakets auf andere Systeme (Word/OpenOffice/LibreOffice, Scribus, \ldots). Hilfe erbeten!
}

\noindent
Ich lerne übrigens selbst gerne dazu. Wenn dir am \LaTeX-Code etwas auffällt, was man besser machen kann, lass' es mich wissen! 

\end{multicols}

\newpage