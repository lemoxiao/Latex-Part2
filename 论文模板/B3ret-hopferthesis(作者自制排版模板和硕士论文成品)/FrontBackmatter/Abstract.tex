%*******************************************************
% Abstract
%*******************************************************
%\renewcommand{\abstractname}{Abstract}
\pdfbookmark[1]{Abstract}{Abstract}
\begingroup
\let\clearpage\relax
\let\cleardoublepage\relax
\let\cleardoublepage\relax


\chapter*{Abstract}
Results from computational fluid dynamics (\acs{CFD}) simulations are generally complex and difficult to understand.
This work proposes a new method that computes from a given simulation result, \eg, the underhood flow of air around a car engine, a sparse directed graph network with a few hundred nodes.
The goal is a graph that preserves the essential properties of the flow in such way that it is suitable for  applications ranging from information visualization to flow simulation.
The algorithm finds bundles of similar streamline segments, which are then mapped back to the original dataset in order to produce a complete partition.
A flow graph is derived from this partition by integration over the \acs{CFD} cells.
By utilizing a custom-built simulation framework, the proposed method is shown to produce meaningful graphs, which can be used within the mentioned application areas.


\vfill

\begin{otherlanguage}{ngerman}
%
\pdfbookmark[1]{Zusammenfassung}{Zusammenfassung}
\chapter*{Zusammenfassung}

Ergebnisse von numerischen Flusssimulationen (computational fluid dynamics, \acs{CFD}) sind im Allgemeinen komplex und schwierig zu verstehen. Diese Arbeit stellt eine neue Methode vor. Ausgehend vom Ergebnis einer Str�mungssimulation, beispielsweise von Luft im Motorraum eines Autos, wird ein d�nner, gerichteter Graph mit wenigen hundert Knoten erzeugt.
Ziel der Arbeit ist ein Graph, welcher die essenziellen Flusseigenschaften in einer Weise abbildet, die Anwendungen von der Visualisierung bis hin zur Simulation erlaubt.
Der Algorithmus findet B�ndel �hnlicher Stromliniensegmente und ordnet diese dann wieder dem urspr�nglichen Datensatz zu, um eine vollst�ndige Partition zu erzeugen. 
Aus dieser Partition entsteht durch Integration �ber die \acs{CFD}-Zellen ein Flussgraph.
Durch Einsatz einer speziell implementierten Simulationsumgebung wird gezeigt, dass die vorgeschlagene Methode sinnvolle Graphen erzeugt, welche in den erw�hnten Anwendungsgebieten verwendet werden k�nnen.
%
\end{otherlanguage}


\endgroup			

\vfill